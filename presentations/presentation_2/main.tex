\documentclass[14pt, aspectratio=169]{beamer}
\usepackage{unicode}
\usepackage[notocbib]{apacite}
\usepackage{parskip}
\usepackage{times}
\usepackage{tabularx}
\usepackage{mdframed}
\usepackage{caption}
\usepackage{xcolor}
\usepackage{pifont}
\usepackage{multicol}

\newcommand {\scaledimage}[1] {
    \includegraphics[width=\textwidth,height=0.8\textheight,keepaspectratio]{#1}
}

\renewcommand{\bibname}{Referencias Bibliográficas}
\renewcommand{\contentsname}{Tabla de Contenido}
\renewcommand{\chaptername}{Capítulo}
\renewcommand{\figurename}{Figura}
\renewcommand{\tablename}{Tabla}

\title{Implementación del Analizador Léxico}
\author{Aramis E. Matos, Lenier Gerena, Angel Berrios}
\date{3/30/2023}

\usetheme{Warsaw}
% \usecolortheme{beaver}
\setbeamertemplate{headline}{}
\setbeamertemplate{navigation symbols}{}

\begin{document}
\maketitle

\begin{frame}
    \frametitle{Introducción}
    \begin{columns}
        \column{0.5\textwidth}
        \small
        \begin{itemize}
            \item El propósito de esta presentación es exponer acerca de la implementación léxica del lenguaje AVISMO.
            \item La implementación fue inspirada por la documentación de Python Lex Yacc (PLY)  \cite{noauthor_ply_nodate} y el analizador léxico de MAPL \cite{noauthor_pl-project-lgm-yvv-amnmapl_nodate}
        \end{itemize}
        \column{0.5\textwidth}
        \begin{itemize}
            \item El analizador léxico utilizado fue PLY
        \end{itemize}
    \end{columns}
\end{frame}


\begin{frame}
    \frametitle{Referencias}
    \bibliography{../../report/References}
    \bibliographystyle{apacite}
\end{frame}

\end{document}
