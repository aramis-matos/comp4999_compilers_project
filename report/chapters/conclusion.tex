Inicialmente, el analizador léxico de este proyecto fue escrito con \textit{Flex} \cite{noauthor_flex_nodate} en \textit{C++}. 
Esto se hizo porque nosotros no habíamos escrito en \textit{C++} en mucho tiempo y deseábamos elaborar un proyecto extenso en él para mejorar nuestro entendimiento del lenguaje. 
Sin embargo, se tuvo que abandonar este camino debido a una combinación de limitaciones de tiempo, documentación pobre, pocos recursos de donde tomar inspiración, etc. 
Debido a esta situación, se tuvo que re-escribir el analizador léxico en PLY, la cual tiene documentación mejor, recursos extensos, buenos ejemplos, entre otros beneficios. 
De este cambio se aprendieron varias lecciones. 
Entre ellas, la importancia de utilizar la herramienta más apropiada para el trabajo. 
En el desarrollo de \textit{software}, es importante utilizar herramientas que tengan una base amplia de soporte, independientemente de las metas personales de los diseñadores. 

En conclusión, este proyecto fue una experiencia fascinante y despertadora.
Como programadores, los compiladores y interpretadores son nuestras herramientas de uso diario, como es el martillo para un carpintero.
Aveces se nos olvida que los lenguaje de programación están diseñados para ser escritos y entendidos por humanos porque su estructura parece tan disimilar a las lenguas naturales. 
Gracias a esta experiencia, dimos un paso hacia atrás y pudimos apreciar lo complejo que es diseñar un analizador léxico y sintáctico.
Creemos que jamás perderemos la paciencia con un error de compilación.
Sinó nos sentiremos agradecidos que algún programador tomo el tiempo de crear la herramientas que nos provee un sueldo. 
No creemos que sea una exageración decir que gracias a la labor colaborativa de muchos académicos a traves del tiempo, han cambiado el mundo, una línea de código a la vez.