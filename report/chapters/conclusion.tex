Inicialmente, el analizador léxico de este proyecto fue escrito con \textit{Flex} \cite{noauthor_flex_nodate} en \textit{C++}. 
Esto se hizo porque nosotros no habíamos escrito en \textit{C++} en mucho tiempo y deseábamos elaborar un proyecto extenso en el para mejorar nuestro entendimiento del lenguaje. 
Sin embargo, se tuvo que abandonar este camino debido a una combinación de limitaciones de tiempo, documentación pobre, pocos recursos de donde tomar inspiración, etc. 
Debido a esta situación se tuvo que re-escribir el analizador léxico en \textit{PLY}, la cual tiene documentación mejor, recursos extensos, buen ejemplos, entre otros beneficios. 
De este cambio se aprendieron varias lecciones. 
Entre ellas la importancia de utilizar la herramienta mas apropiada para el trabajo. 
En el desarrollo de \textit{software}, es importante utilizar herramientas que tengan una base amplia de soporte, independientemente de las metas personales de los diseñadores. 

En conclusión, este proyecto fue una experiencia fascinante y despertadora.
Como programadores, los compiladores y interpretadores son nuestras herramientas de uso diario, como es el martillo para un carpintero.
Aveces se nos olvida que los lenguaje de programación están diseñadas para ser escritos y entendidos por humanos porque su estructura parece tan disimilar a las lenguas naturales. 
Gracias a esta experiencia, dimos un paso hacia atrás y pudimos apreciar lo complejo que es diseñar un analizador léxico y sintáctico.
Creemos que jamas perderemos la paciencia con un error de compilación.
Sino nos sentiremos agradecidos algún programador tomo el tiempo de crear la herramientas que no tan solo nos provee un sueldo. 
No creemos que sea una exageración decir que gracias a la labor colaborativa de muchos académicos a traves del tiempo, han cambiado el mundo, una línea de código a la vez.