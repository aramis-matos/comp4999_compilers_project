\section{Gramatica del Lenguaje AVISMO}

\begin{itemize}
    \item <SENTENCIAS> ::= <FIN\_DE\_LINEA> <SENTENCIAS> | <SENTENCIA> <FIN\_DE\_LINEA>
    \item <FIN\_DE\_LINEA> ::= ":" | ";"
    \item <SENTENCIA> ::= "defina" <ID> "como" <TIPO> | <ID> "="  <MODELO\_MOLECULAR> | <OPERACION> "(" <ID> ")"
    \item <ID> ::= "A" | "B" | "C" | "D" | "E" | "F" | "G" | "H" | "I" | "J" | "K" | "L" | "M" | "N" | "O" | "P" | "Q" | "R" | "S" | "T" | "U" | "V" | "W" | "X" | "Y" | "Z" | "a" | "b" | "c" | "d" | "e" | "f" | "g" | "h" | "i" | "j" | "k" | "l" | "m" | "n" | "o" | "p" | "q" | "r" | "s" | "t" | "u" | "v" | "w" | "x" | "y" | "z" | <LETRA> <IDCONT>
    \item <IDCONT> ::= "A" | "B" | "C" | "D" | "E" | "F" | "G" | "H" | "I" | "J" | "K" | "L" | "M" | "N" | "O" | "P" | "Q" | "R" | "S" | "T" | "U" | "V" | "W" | "X" | "Y" | "Z" | "a" | "b" | "c" | "d" | "e" | "f" | "g" | "h" | "i" | "j" | "k" | "l" | "m" | "n" | "o" | "p" | "q" | "r" | "s" | "t" | "u" | "v" | "w" | "x" | "y" | "z" | <LETRA> <IDCONT> | "0" | "1" | "2" | "3" | "4" | "5" | "6" | "7" | "8" | "9" | <DIGITO> <IDCONT>
    \item <LETRA> ::= "A" | "B" | "C" | "D" | "E" | "F" | "G" | "H" | "I" | "J" | "K" | "L" | "M" | "N" | "O" | "P" | "Q" | "R" | "S" | "T" | "U" | "V" | "W" | "X" | "Y" | "Z" | "a" | "b" | "c" | "d" | "e" | "f" | "g" | "h" | "i" | "j" | "k" | "l" | "m" | "n" | "o" | "p" | "q" | "r" | "s" | "t" | "u" | "v" | "w" | "x" | "y" | "z"
    \item <DIGITO> ::= "0" | "1" | "2" | "3" | "4" | "5" | "6" | "7" | "8" | "9"
    \item <TIPO> ::= "modelo"
    \item <OPERACION> ::= "graficar2d" |"graficar3d" | "pesomolecular"
    \item <MODELO\_MOLECULAR> ::= "H" | "Li" | "Na" | "K" | "Rb" | "Cs" | "Fr" | "Be" | "Mg" | "Ca" | "Sr" | "Ba" | "Ra" | "Sc" | "Y" | "Ti" | "Zr" | "Hf" | "Db" | "V" | "Nb" | "Ta" | "Ji" | "Cr" | "Mo" | "W" | "Rf" | "Mn" | "Tc" | "Re" | "Bh" | "Fe" | "Ru" | "Os" | "Hn" | "Co" | "Rh" | "Ir" | "Mt" | "Ni" | "Pd" | "Pt" | "Cu" | "Ag" | "Au" | "Zn" | "Cd" | "Hg" | "B" | "Al" | "Ga" | "In" | "Ti" | "C" | "Si" | "Ge" | "Sn" | "Pb" | "N" | "P" | "As" | "Sb" | "Bi" | "O" | "S" | "Se" | "Te" | "Po" | "F" | "Cr" | "Br" | "I" | "At" | "He" | "Ne" | "Ar" | "Kr" | "Xe" | "Rn" | <ELEMENTO\_QUIMICO> <VALENCIA> | <ELEMENTO> <GRUPO\_FUNCIONAL> | <COMPUESTO> <ELEMENTO> <GRUPO\_FUNCIONAL> | <COMPUESTO> <COMPUESTO>
    \item <COMPUESTO> ::= "H" | "Li" | "Na" | "K" | "Rb" | "Cs" | "Fr" | "Be" | "Mg" | "Ca" | "Sr" | "Ba" | "Ra" | "Sc" | "Y" | "Ti" | "Zr" | "Hf" | "Db" | "V" | "Nb" | "Ta" | "Ji" | "Cr" | "Mo" | "W" | "Rf" | "Mn" | "Tc" | "Re" | "Bh" | "Fe" | "Ru" | "Os" | "Hn" | "Co" | "Rh" | "Ir" | "Mt" | "Ni" | "Pd" | "Pt" | "Cu" | "Ag" | "Au" | "Zn" | "Cd" | "Hg" | "B" | "Al" | "Ga" | "In" | "Ti" | "C" | "Si" | "Ge" | "Sn" | "Pb" | "N" | "P" | "As" | "Sb" | "Bi" | "O" | "S" | "Se" | "Te" | "Po" | "F" | "Cr" | "Br" | "I" | "At" | "He" | "Ne" | "Ar" | "Kr" | "Xe" | "Rn" | <ELEMENTO\_QUIMICO> <VALENCIA> | <ELEMENTO> <GRUPO\_FUNCIONAL> | <ELEMENTO> <GRUPO\_FUNCIONAL> <ENLACE> | <ELEMENTO> <ENLACE>
    \item <COMPUESTOS> ::= <COMPUESTO> <COMPUESTO> | <COMPUESTOS>
    \item <ELEMENTO> ::= "H" | "Li" | "Na" | "K" | "Rb" | "Cs" | "Fr" | "Be" | "Mg" | "Ca" | "Sr" | "Ba" | "Ra" | "Sc" | "Y" | "Ti" | "Zr" | "Hf" | "Db" | "V" | "Nb" | "Ta" | "Ji" | "Cr" | "Mo" | "W" | "Rf" | "Mn" | "Tc" | "Re" | "Bh" | "Fe" | "Ru" | "Os" | "Hn" | "Co" | "Rh" | "Ir" | "Mt" | "Ni" | "Pd" | "Pt" | "Cu" | "Ag" | "Au" | "Zn" | "Cd" | "Hg" | "B" | "Al" | "Ga" | "In" | "Ti" | "C" | "Si" | "Ge" | "Sn" | "Pb" | "N" | "P" | "As" | "Sb" | "Bi" | "O" | "S" | "Se" | "Te" | "Po" | "F" | "Cr" | "Br" | "I" | "At" | "He" | "Ne" | "Ar" | "Kr" | "Xe" | "Rn" | <ELEMENTO\_QUIMICO> <VALENCIA>
    \item <ELEMENTO\_QUIMICO> ::= "H" | "Li" | "Na" | "K" | "Rb" | "Cs" | "Fr" | "Be" | "Mg" | "Ca" | "Sr" | "Ba" | "Ra" | "Sc" | "Y" | "Ti" | "Zr" | "Hf" | "Db" | "V" | "Nb" | "Ta" | "Ji" | "Cr" | "Mo" | "W" | "Rf" | "Mn" | "Tc" | "Re" | "Bh" | "Fe" | "Ru" | "Os" | "Hn" | "Co" | "Rh" | "Ir" | "Mt" | "Ni" | "Pd" | "Pt" | "Cu" | "Ag" | "Au" | "Zn" | "Cd" | "Hg" | "B" | "Al" | "Ga" | "In" | "Ti" | "C" | "Si" | "Ge" | "Sn" | "Pb" | "N" | "P" | "As" | "Sb" | "Bi" | "O" | "S" | "Se" | "Te" | "Po" | "F" | "Cr" | "Br" | "I" | "At" | "He" | "Ne" | "Ar" | "Kr" | "Xe" | "Rn"
    \item <VALENCIA> ::= "1" | "2" | "3" | "4" | "5" | "6" | "7" | "8" | "9"
    \item <GRUPO\_FUNCIONAL> ::=
          <GRUPO\_FUNCIONAL\_INFERIOR>

          <GRUPO\_FUNCIONAL\_SUPERIOR> | <GRUPO\_FUNCIONAL\_SUPERIOR> <GRUPO\_FUNCIONAL\_INFERIOR> | "(" <MODELO\_GRUPO\_FUNCIONAL> ")" | "["  <MODELO\_GRUPO\_FUNCIONAL> "]"
    \item <GRUPO\_FUNCIONAL\_SUPERIOR> ::= "[" <MODELO\_GRUPO\_FUNCIONAL> "]"
    \item <GRUPO\_FUNCIONAL\_INFERIOR> ::= "(" <MODELO\_GRUYPO\_FUNCIONAL> ")
    \item  <MODELO\_GRUYPO\_FUNCIONAL> ::= <ENLACE> <MODELO\_MOLECULAR> | "H" | "Li" | "Na" | "K" | "Rb" | "Cs" | "Fr" | "Be" | "Mg" | "Ca" | "Sr" | "Ba" | "Ra" | "Sc" | "Y" | "Ti" | "Zr" | "Hf" | "Db" | "V" | "Nb" | "Ta" | "Ji" | "Cr" | "Mo" | "W" | "Rf" | "Mn" | "Tc" | "Re" | "Bh" | "Fe" | "Ru" | "Os" | "Hn" | "Co" | "Rh" | "Ir" | "Mt" | "Ni" | "Pd" | "Pt" | "Cu" | "Ag" | "Au" | "Zn" | "Cd" | "Hg" | "B" | "Al" | "Ga" | "In" | "Ti" | "C" | "Si" | "Ge" | "Sn" | "Pb" | "N" | "P" | "As" | "Sb" | "Bi" | "O" | "S" | "Se" | "Te" | "Po" | "F" | "Cr" | "Br" | "I" | "At" | "He" | "Ne" | "Ar" | "Kr" | "Xe" | "Rn" | <ELEMENTO\_QUIMICO> <VALENCIA> | <ELEMENTO> <GRUPO\_FUNCIONAL> | <COMPUESTO> <ELEMENTO> | <COMPUESTO> <COMPUESTO> <COMPUESTOS>
\end{itemize}
\vspace{1cm}
En la tabla \ref{table:lexTable}, en la columna de patrones, note que cuando dice $\left\{\textit{TOKEN}\right\}$ donde \textit{TOKEN} se refiere a el patrón asociado a \textit{token}.
Por ejemplo, si un patrón dice $\left\{\textit{ELEMENTO\_QUIMICO}\right\}$, esto significa que inserta el patrón asociado al \textit{token} \textit{ELEMENTO\_QUIMICO}.
Esto no significa que el analizador léxico espera un \textit{token} de por si, sencillamente se hizo con el propósito de evitar redundancias.
\newpage

\begin{landscape}
    \footnotesize
    \begin{longtable}{| p{0.2\textheight} | p{0.75\textheight} | p{0.2\textheight} | p{0.25\textheight} |}
        \hline
        \textit{Token}                 & Patrón                                                                                                                                                                                                                                                                                                                                                                                                                                                                                                                                                       & Lexema        & Atributos                                                              \\\hline
        <FIN\_DE\_LINEA>               & ; | :                                                                                                                                                                                                                                                                                                                                                                                                                                                                                                                                                        & :             & Simbolo reservado                                                      \\\hline
        <PALABRA \_RESERVADA>          & defina | como                                                                                                                                                                                                                                                                                                                                                                                                                                                                                                                                                & defina        & Palabra reservada                                                      \\\hline
        <ID>                           & [A-Za-z][A-Za-z0-9]*                                                                                                                                                                                                                                                                                                                                                                                                                                                                                                                                         & var1          & Modelo molecular asociado                                              \\\hline
        <IDCONT>                       & \textit{[A-Za-z0-9]+}                                                                                                                                                                                                                                                                                                                                                                                                                                                                                                                                        & 1ar           & ID asociado                                                            \\\hline
        <LETRA>                        & [A-Za-z]                                                                                                                                                                                                                                                                                                                                                                                                                                                                                                                                                     & a             & ID asociado                                                            \\\hline
        <DIGITO>                       & [0-9]                                                                                                                                                                                                                                                                                                                                                                                                                                                                                                                                                        & 7             & Valor numérico, lexema asociado                                        \\\hline
        <TIPO>                         & modelo                                                                                                                                                                                                                                                                                                                                                                                                                                                                                                                                                       & modelo        & ID asociado                                                            \\\hline
        <OPERACION>                    & graficar2d | graficar3d | pesomolecular                                                                                                                                                                                                                                                                                                                                                                                                                                                                                                                      & pesomolecular & ID asociado                                                            \\\hline
        <MODELO \_MOLECULAR>           & (\{ELEMENTO \_QUIMICO\} | \{ELEMENTO \_QUIMICO\} \{VALENCIA\} | \{ELEMENTO\} \{GRUPO \_FUNCIONAL\} | \{ELEMENTO\} \{GRUPO \_FUNCIONAL\} \{ENLACE\} | \{ELEMENTO\} \{ENLACE\})                                                                                                                                                                                                                                                                                                                                                                                & CH3(CH3)CHH   & ID asociado                                                            \\\hline
        <COMPUESTO>                    & COMPUESTO (\{ELEMENTO \_QUIMICO\} | \{ELEMENTO \_QUIMICO\} \{VALENCIA\} | \{ELEMENTO\} \{GRUPO\_FUNCIONAL\} | \{ELEMENTO\} \{GRUPO \_FUNCIONAL\} \{ENLACE\} | \{ELEMENTO\} \{ENLACE\})                                                                                                                                                                                                                                                                                                                                                                       & CH3::         & Modelo molecular asociado, enlaces, valencias                          \\\hline
        <COMPUESTOS>                   & \{COMPUESTO\}+                                                                                                                                                                                                                                                                                                                                                                                                                                                                                                                                               & CH3::(OH)3    & Modelo molecular asociado, enlaces, valencias                          \\\hline
        <ELEMENTO>                     & \{ELEMENTO \_QUIMICO\} \{VALENCIA\}?                                                                                                                                                                                                                                                                                                                                                                                                                                                                                                                         & Ag3           & Elemento, valencia                                                     \\\hline
        <ELEMENTO \_QUIMICO>           & ( "H" | "Li" | "Na" | "K" | "Rb" | "Cs" | "Fr" | "Be" | "Mg" | "Ca" | "Sr" | "Ba" | "Ra" | "Sc" | "Y" | "Ti" | "Zr" | "Hf" | "Db" | "V" | "Nb" | "Ta" | "Ji" | "Cr" | "Mo" | "W" | "Rf" | "Mn" | "Tc" | "Re" | "Bh" | "Fe" | "Ru" | "Os" | "Hn" | "Co" | "Rh" | "Ir" | "Mt" | "Ni" | "Pd" | "Pt" | "Cu" | "Ag" | "Au" | "Zn" | "Cd" | "Hg" | "B" | "Al" | "Ga" | "In" | "Ti" | "C" | "Si" | "Ge" | "Sn" | "Pb" | "N" | "P" | "As" | "Sb" | "Bi" | "O" | "S" | "Se" | "Te" | "Po" | "F" | "Cr" | "Br" | "I" | "At" | "He" | "Ne" | "Ar" | "Kr" | "Xe" | "Rn") & I             & Elemento                                                               \\\hline
        <VALENCIA>                     & [1-9]                                                                                                                                                                                                                                                                                                                                                                                                                                                                                                                                                        & 2             & Valor                                                                  \\\hline
        <GRUPO \_FUNCIONAL>            & ( \{GRUPO \_FUNCIONAL \_INFERIOR\} \{GRUPO \_FUNCIONAL \_SUPERIOR\} | \{GRUPO \_FUNCIONAL \_SUPERIOR\} \{GRUPO \_FUNCIONAL\_INFERIOR\} | "(" \{MODELO \_GRUPO \_FUNCIONAL\} ")" | "[" {MODELO \_GRUPO \_FUNCIONAL} "]")                                                                                                                                                                                                                                                                                                                                      & (CH3)\{Ag2\}  & Grupos funcionales, grupo funcional inferior, grupo funcional superior \\\hline
        <GRUPO \_FUNCIONAL \_INFERIOR> & "[" \{MODELO \_GRUPO \_FUNCIONAL\} "]"                                                                                                                                                                                                                                                                                                                                                                                                                                                                                                                       & [CVHe3]       & Elementos, valencias                                                   \\\hline
        <GRUPO \_FUNCIONAL \_SUPERIOR> & "(" \{MODELO \_GRUPO \_FUNCIONAL\} ")"                                                                                                                                                                                                                                                                                                                                                                                                                                                                                                                       & (CVHe3)       & Elementos, valencias                                                   \\\hline
        <MODELO \_GRUPO \_FUNCIONAL>   & (\{ELEMENTO \_QUIMICO\}+ \{VALENCIA\}?)+ | (\{ELEMENTO\}+ \{ENLACE\} \{ELEMENTO\}+)+                                                                                                                                                                                                                                                                                                                                                                                                                                                                         & FeH=C3Si4     & Elementos, enlaces, valencias                                          \\\hline
        <ENLACE>                       & ("-" | "=" | ":" | "::")                                                                                                                                                                                                                                                                                                                                                                                                                                                                                                                                     & -             & Valencia                                                               \\\hline
        \caption{Tabla de Componentes Léxicos de AVISMO}
        \label{table:lexTable}
    \end{longtable}
\end{landscape}

\section{Diseño del del Analizador Léxico}

\subsection{Autómatas Finitos Deterministas}

\begin{figure}[H]
    \footnotesize
    \begin{minipage}{.5\textwidth}
        \centering
        \begin{mdframed}
            \begin{tikzpicture}[node distance = 2.5cm, on grid, auto]
                \node [state,initial] (q0) {$0$};
                \node [state,accepting] [right=of q0] (q1) {$1$};
                \node [state,accepting] [below=of q1] (q2) {$2$};

                \path[-stealth, thick]
                (q0) edge node {;} (q1)
                (q0) edge [bend right] node {:} (q2);
            \end{tikzpicture}
            \label{fig: finDeLineaAutomata}
        \end{mdframed}
        \caption{Automata del patrón para el token <FIN\_DE\_LINEA>}
    \end{minipage}\hspace{1cm}
    \begin{minipage}{0.5\textwidth}
        \centering
        \begin{mdframed}
            \begin{tikzpicture}[node distance = 2.5cm, on grid, auto]
                \node [state, initial] (q0) {$0$};
                \node [state, accepting] [right=of q0] (q1) {$1$};
                \node [state, accepting] [below=of q1] {$1$};

                \path[-stealth, thick]
                (q0) edge node {defina} (q1)
                (q0) edge [bend right] node {como} (q2);
            \end{tikzpicture}
        \end{mdframed}
        \label{fig: palabraReservada}
        \captionof{figure}{Automata del patrón para el token <PALABRAS\_RESERVADA>}
    \end{minipage}
\end{figure}


\begin{figure}[H]
    \footnotesize
    \begin{minipage}{0.5\textwidth}
        \begin{mdframed}
            \begin{tikzpicture}[node distance = 2.5cm, auto]
                \node [state, initial] (q0) {0};
                \node [state, accepting] [right=of q0] (q1) {$1$};
                \node [state] [below=of q0] (q2) {$2$};
                \node [state, accepting] [right=of q2] (q3) {$3$};

                \path[-stealth, thick]
                (q0) edge node {[A-Za-z]} (q1)
                (q0) edge node {<LETRA>} (q2)
                (q2) edge node {<IDCONT>} (q3);
            \end{tikzpicture}
        \end{mdframed}
        \label{fig: idAutomata}
        \caption{Automata del patrón para el token <ID>}
    \end{minipage}\hspace{1cm}
    \begin{minipage}{0.5\textwidth}
        \begin{mdframed}
            \begin{tikzpicture}[node distance = 1.5cm, auto]
                \node [state, initial] (q0) {$0$};
                \node [state, accepting] [right=of q0] (q1) {$1$};
                \node [state] [below=of q1] (q2) {$2$};
                \node [state, accepting] [above left=of q0] (q4) {$4$};
                \node [state] [below left=of q0] (q5) {$5$};


                \path[-stealth, thick]
                (q0) edge node {[A-Za-z]} (q1)
                (q0) edge node [left] {<LETRA>} (q2)
                (q2) edge [bend right] node [right]{<IDCONT>} (q0)
                (q0) edge [bend right] node[right] {[0-9]} (q4)
                (q0) edge [bend right] node [left] {<DIGITO>} (q5)
                (q5.east) edge [bend right] node [below] {<IDCONT>} (q0);


            \end{tikzpicture}
        \end{mdframed}
        \label{fig: idContAutomata}
        \caption{Automata del patrón para el token <IDCONT>}
    \end{minipage}
\end{figure}

\begin{figure}[H]
    \footnotesize
    \begin{minipage}{0.5\textwidth}
        \begin{mdframed}
            \begin{tikzpicture}[node distance = 1.5cm, on grid, auto]
                \node [state,initial] (q0) {$0$};
                \node [state,accepting] [right=of q1] (q1) {$1$};

                \path[-stealth,thick]
                (q0) edge node {=} (q1);
            \end{tikzpicture}
        \end{mdframed}
        \label{fig: asigAutomata}
        \caption{Automata del patrón para el token <ASIGNACION>}
    \end{minipage}\hspace{1cm}
    \begin{minipage}{0.5\linewidth}
        \begin{mdframed}
            \begin{tikzpicture}[node distance = 0cm, on grid ,auto]
                \node [state,initial] (q0) {$0$};
                \node [state,accepting] [right=of q1] (q1) {$1$};

                \path[-stealth,thick]
                (q0) edge node {[0-9]} (q1);
            \end{tikzpicture}
        \end{mdframed}
        \label{fig: letraAutomata}
        \caption{Automata del patrón para el token <LETRA>}
    \end{minipage}
\end{figure}

\begin{center}
    \begin{figure}[H]
        \footnotesize
        \begin{mdframed}
            \begin{tikzpicture}[node distance = 0cm, on grid ,auto]
                \node [state,initial] (q0) {$0$};
                \node [state,accepting] [right=of q1] (q1) {$1$};

                \path[-stealth,thick]
                (q0) edge node {[0-9]} (q1);
            \end{tikzpicture}
        \end{mdframed}
        \label{fig: digitoAutomata}
        \caption{Automata del patrón para el token <DIGITO>}
    \end{figure}
\end{center}

\subsection{Tabla de Símbolos}

\begin{figure}[H]
    \begin{tikzpicture}[
            node distance=2cm and 3cm,
            ID/.style={rectangle, rounded corners, draw=black, thick, minimum size=10mm},
            ATT/.style={rectangle, rounded corners, draw=red!60, thick, minimum size=10mm},
        ]
        \draw node at (0,1.6)   {Identificador};
        \draw node at (7.5, 1.6)  {Atributo};

        \node[ID]   (var1)                  {var1};
        \node[ID]   (Big)   [below=of var1] {Big};

        \node[ATT]  (str1)  [right=of var1]  {str, val:"chungus", esMutable:false};
        \node[ATT]  (int1)  [right=of Big]   {int, val:"12",      esMutable:false};

        \draw[->, very thick] (var1.east) to (str1.west);
        \draw[->, very thick] (Big.east) to (int1.west);

        \node[draw=black, thick, fit={(var1) (Big)}, inner sep=10pt] (box) {};
        \node[draw=black, thick, fit={(str1) (int1)}, inner sep=10pt] (box) {};
    \end{tikzpicture}
    \label{fig: tablaDeSimbolos}
    \caption{Tabla de símbolos implementada como un diccionario}
\end{figure}

\section{Implementación del Analizador Léxico}