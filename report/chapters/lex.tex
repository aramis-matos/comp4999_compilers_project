\section{Gramatica del Lenguaje AVISMO}

\begin{itemize}
    \item <SENTENCIAS> ::= <FIN\_DE\_LINEA> <SENTENCIAS> | <SENTENCIA> <FIN\_DE\_LINEA>
    \item <FIN\_DE\_LINEA> ::= ":" | ";"
    \item <SENTENCIA> ::= "defina" <ID> "como" <TIPO> | <ID> "="  <MODELO\_MOLECULAR> | <OPERACION> "(" <ID> ")"
    \item <ID> ::= "A" | "B" | "C" | "D" | "E" | "F" | "G" | "H" | "I" | "J" | "K" | "L" | "M" | "N" | "O" | "P" | "Q" | "R" | "S" | "T" | "U" | "V" | "W" | "X" | "Y" | "Z" | "a" | "b" | "c" | "d" | "e" | "f" | "g" | "h" | "i" | "j" | "k" | "l" | "m" | "n" | "o" | "p" | "q" | "r" | "s" | "t" | "u" | "v" | "w" | "x" | "y" | "z" | <LETRA> <IDCONT>
    \item <IDCONT> ::= "A" | "B" | "C" | "D" | "E" | "F" | "G" | "H" | "I" | "J" | "K" | "L" | "M" | "N" | "O" | "P" | "Q" | "R" | "S" | "T" | "U" | "V" | "W" | "X" | "Y" | "Z" | "a" | "b" | "c" | "d" | "e" | "f" | "g" | "h" | "i" | "j" | "k" | "l" | "m" | "n" | "o" | "p" | "q" | "r" | "s" | "t" | "u" | "v" | "w" | "x" | "y" | "z" | <LETRA> <IDCONT> | "0" | "1" | "2" | "3" | "4" | "5" | "6" | "7" | "8" | "9" | <DIGITO> <IDCONT>
    \item <LETRA> ::= "A" | "B" | "C" | "D" | "E" | "F" | "G" | "H" | "I" | "J" | "K" | "L" | "M" | "N" | "O" | "P" | "Q" | "R" | "S" | "T" | "U" | "V" | "W" | "X" | "Y" | "Z" | "a" | "b" | "c" | "d" | "e" | "f" | "g" | "h" | "i" | "j" | "k" | "l" | "m" | "n" | "o" | "p" | "q" | "r" | "s" | "t" | "u" | "v" | "w" | "x" | "y" | "z"
    \item <DIGITO> ::= "0" | "1" | "2" | "3" | "4" | "5" | "6" | "7" | "8" | "9"
    \item <TIPO> ::= "modelo"
    \item <OPERACION> ::= "graficar2d" |"graficar3d" | "pesomolecular"
    \item <MODELO\_MOLECULAR> ::= "H" | "Li" | "Na" | "K" | "Rb" | "Cs" | "Fr" | "Be" | "Mg" | "Ca" | "Sr" | "Ba" | "Ra" | "Sc" | "Y" | "Ti" | "Zr" | "Hf" | "Db" | "V" | "Nb" | "Ta" | "Ji" | "Cr" | "Mo" | "W" | "Rf" | "Mn" | "Tc" | "Re" | "Bh" | "Fe" | "Ru" | "Os" | "Hn" | "Co" | "Rh" | "Ir" | "Mt" | "Ni" | "Pd" | "Pt" | "Cu" | "Ag" | "Au" | "Zn" | "Cd" | "Hg" | "B" | "Al" | "Ga" | "In" | "Ti" | "C" | "Si" | "Ge" | "Sn" | "Pb" | "N" | "P" | "As" | "Sb" | "Bi" | "O" | "S" | "Se" | "Te" | "Po" | "F" | "Cr" | "Br" | "I" | "At" | "He" | "Ne" | "Ar" | "Kr" | "Xe" | "Rn" | <ELEMENTO\_QUIMICO> <VALENCIA> | <ELEMENTO> <GRUPO\_FUNCIONAL> | <COMPUESTO> <ELEMENTO> <GRUPO\_FUNCIONAL> | <COMPUESTO> <COMPUESTO>
    \item <COMPUESTO> ::= "H" | "Li" | "Na" | "K" | "Rb" | "Cs" | "Fr" | "Be" | "Mg" | "Ca" | "Sr" | "Ba" | "Ra" | "Sc" | "Y" | "Ti" | "Zr" | "Hf" | "Db" | "V" | "Nb" | "Ta" | "Ji" | "Cr" | "Mo" | "W" | "Rf" | "Mn" | "Tc" | "Re" | "Bh" | "Fe" | "Ru" | "Os" | "Hn" | "Co" | "Rh" | "Ir" | "Mt" | "Ni" | "Pd" | "Pt" | "Cu" | "Ag" | "Au" | "Zn" | "Cd" | "Hg" | "B" | "Al" | "Ga" | "In" | "Ti" | "C" | "Si" | "Ge" | "Sn" | "Pb" | "N" | "P" | "As" | "Sb" | "Bi" | "O" | "S" | "Se" | "Te" | "Po" | "F" | "Cr" | "Br" | "I" | "At" | "He" | "Ne" | "Ar" | "Kr" | "Xe" | "Rn" | <ELEMENTO\_QUIMICO> <VALENCIA> | <ELEMENTO> <GRUPO\_FUNCIONAL> | <ELEMENTO> <GRUPO\_FUNCIONAL> <ENLACE> | <ELEMENTO> <ENLACE>
    \item <COMPUESTOS> ::= <COMPUESTO> <COMPUESTO> | <COMPUESTOS>
    \item <ELEMENTO> ::= "H" | "Li" | "Na" | "K" | "Rb" | "Cs" | "Fr" | "Be" | "Mg" | "Ca" | "Sr" | "Ba" | "Ra" | "Sc" | "Y" | "Ti" | "Zr" | "Hf" | "Db" | "V" | "Nb" | "Ta" | "Ji" | "Cr" | "Mo" | "W" | "Rf" | "Mn" | "Tc" | "Re" | "Bh" | "Fe" | "Ru" | "Os" | "Hn" | "Co" | "Rh" | "Ir" | "Mt" | "Ni" | "Pd" | "Pt" | "Cu" | "Ag" | "Au" | "Zn" | "Cd" | "Hg" | "B" | "Al" | "Ga" | "In" | "Ti" | "C" | "Si" | "Ge" | "Sn" | "Pb" | "N" | "P" | "As" | "Sb" | "Bi" | "O" | "S" | "Se" | "Te" | "Po" | "F" | "Cr" | "Br" | "I" | "At" | "He" | "Ne" | "Ar" | "Kr" | "Xe" | "Rn" | <ELEMENTO\_QUIMICO> <VALENCIA>
    \item <ELEMENTO\_QUIMICO> ::= "H" | "Li" | "Na" | "K" | "Rb" | "Cs" | "Fr" | "Be" | "Mg" | "Ca" | "Sr" | "Ba" | "Ra" | "Sc" | "Y" | "Ti" | "Zr" | "Hf" | "Db" | "V" | "Nb" | "Ta" | "Ji" | "Cr" | "Mo" | "W" | "Rf" | "Mn" | "Tc" | "Re" | "Bh" | "Fe" | "Ru" | "Os" | "Hn" | "Co" | "Rh" | "Ir" | "Mt" | "Ni" | "Pd" | "Pt" | "Cu" | "Ag" | "Au" | "Zn" | "Cd" | "Hg" | "B" | "Al" | "Ga" | "In" | "Ti" | "C" | "Si" | "Ge" | "Sn" | "Pb" | "N" | "P" | "As" | "Sb" | "Bi" | "O" | "S" | "Se" | "Te" | "Po" | "F" | "Cr" | "Br" | "I" | "At" | "He" | "Ne" | "Ar" | "Kr" | "Xe" | "Rn"
    \item <VALENCIA> ::= "1" | "2" | "3" | "4" | "5" | "6" | "7" | "8" | "9"
    \item <GRUPO\_FUNCIONAL> ::=
          <GRUPO\_FUNCIONAL\_INFERIOR>

          <GRUPO\_FUNCIONAL\_SUPERIOR> | <GRUPO\_FUNCIONAL\_SUPERIOR> <GRUPO\_FUNCIONAL\_INFERIOR> | "(" <MODELO\_GRUPO\_FUNCIONAL> ")" | "["  <MODELO\_GRUPO\_FUNCIONAL> "]"
    \item <GRUPO\_FUNCIONAL\_SUPERIOR> ::= "[" <MODELO\_GRUPO\_FUNCIONAL> "]"
    \item <GRUPO\_FUNCIONAL\_INFERIOR> ::= "(" <MODELO\_GRUYPO\_FUNCIONAL> ")
    \item  <MODELO\_GRUYPO\_FUNCIONAL> ::= <ENLACE> <MODELO\_MOLECULAR> | "H" | "Li" | "Na" | "K" | "Rb" | "Cs" | "Fr" | "Be" | "Mg" | "Ca" | "Sr" | "Ba" | "Ra" | "Sc" | "Y" | "Ti" | "Zr" | "Hf" | "Db" | "V" | "Nb" | "Ta" | "Ji" | "Cr" | "Mo" | "W" | "Rf" | "Mn" | "Tc" | "Re" | "Bh" | "Fe" | "Ru" | "Os" | "Hn" | "Co" | "Rh" | "Ir" | "Mt" | "Ni" | "Pd" | "Pt" | "Cu" | "Ag" | "Au" | "Zn" | "Cd" | "Hg" | "B" | "Al" | "Ga" | "In" | "Ti" | "C" | "Si" | "Ge" | "Sn" | "Pb" | "N" | "P" | "As" | "Sb" | "Bi" | "O" | "S" | "Se" | "Te" | "Po" | "F" | "Cr" | "Br" | "I" | "At" | "He" | "Ne" | "Ar" | "Kr" | "Xe" | "Rn" | <ELEMENTO\_QUIMICO> <VALENCIA> | <ELEMENTO> <GRUPO\_FUNCIONAL> | <COMPUESTO> <ELEMENTO> | <COMPUESTO> <COMPUESTO> <COMPUESTOS>
\end{itemize}
\vspace{1cm}
En la tabla \ref{table:lexTable}, en la columna de patrones, note que cuando dice $\left\{\textit{TOKEN}\right\}$ donde \textit{TOKEN} se refiere a el patrón asociado a \textit{token}.
Por ejemplo, si un patrón dice $\left\{\textit{ELEMENTO\_QUIMICO}\right\}$, esto significa que inserta el patrón asociado al \textit{token} \textit{ELEMENTO\_QUIMICO}.
Esto no significa que el analizador léxico espera un \textit{token} de por si, sencillamente se hizo con el propósito de evitar redundancias.
\newpage

\begin{landscape}
    \footnotesize
    \begin{longtable}{| p{0.2\textheight} | p{0.75\textheight} | p{0.2\textheight} | p{0.25\textheight} |}
        \hline
        \textit{Token}                 & Patrón                                                                                                                                                                                                                                                                                                                                                                                                                                                                                                                                                       & Lexema        & Atributos                                                              \\\hline
        <FIN\_DE\_LINEA>               & ; | :                                                                                                                                                                                                                                                                                                                                                                                                                                                                                                                                                        & :             & Simbolo reservado                                                      \\\hline
        <PALABRA \_RESERVADA>          & defina | como                                                                                                                                                                                                                                                                                                                                                                                                                                                                                                                                                & defina        & Palabra reservada                                                      \\\hline
        <ID>                           & [A-Za-z][A-Za-z0-9]*                                                                                                                                                                                                                                                                                                                                                                                                                                                                                                                                         & var1          & Modelo molecular asociado                                              \\\hline
        <IDCONT>                       & \textit{[A-Za-z0-9]+}                                                                                                                                                                                                                                                                                                                                                                                                                                                                                                                                        & 1ar           & ID asociado                                                            \\\hline
        <LETRA>                        & [A-Za-z]                                                                                                                                                                                                                                                                                                                                                                                                                                                                                                                                                     & a             & ID asociado                                                            \\\hline
        <DIGITO>                       & [0-9]                                                                                                                                                                                                                                                                                                                                                                                                                                                                                                                                                        & 7             & Valor numérico, lexema asociado                                        \\\hline
        <TIPO>                         & modelo                                                                                                                                                                                                                                                                                                                                                                                                                                                                                                                                                       & modelo        & ID asociado                                                            \\\hline
        <OPERACION>                    & graficar2d | graficar3d | pesomolecular                                                                                                                                                                                                                                                                                                                                                                                                                                                                                                                      & pesomolecular & ID asociado                                                            \\\hline
        <MODELO \_MOLECULAR>           & (\{ELEMENTO \_QUIMICO\} | \{ELEMENTO \_QUIMICO\} \{VALENCIA\} | \{ELEMENTO\} \{GRUPO \_FUNCIONAL\} | \{ELEMENTO\} \{GRUPO \_FUNCIONAL\} \{ENLACE\} | \{ELEMENTO\} \{ENLACE\})                                                                                                                                                                                                                                                                                                                                                                                & CH3(CH3)CHH   & ID asociado                                                            \\\hline
        <COMPUESTO>                    & COMPUESTO (\{ELEMENTO \_QUIMICO\} | \{ELEMENTO \_QUIMICO\} \{VALENCIA\} | \{ELEMENTO\} \{GRUPO\_FUNCIONAL\} | \{ELEMENTO\} \{GRUPO \_FUNCIONAL\} \{ENLACE\} | \{ELEMENTO\} \{ENLACE\})                                                                                                                                                                                                                                                                                                                                                                       & CH3::         & Modelo molecular asociado, enlaces, valencias                          \\\hline
        <COMPUESTOS>                   & \{COMPUESTO\}+                                                                                                                                                                                                                                                                                                                                                                                                                                                                                                                                               & CH3::(OH)3    & Modelo molecular asociado, enlaces, valencias                          \\\hline
        <ELEMENTO>                     & \{ELEMENTO \_QUIMICO\} \{VALENCIA\}?                                                                                                                                                                                                                                                                                                                                                                                                                                                                                                                         & Ag3           & Elemento, valencia                                                     \\\hline
        <ELEMENTO \_QUIMICO>           & ( "H" | "Li" | "Na" | "K" | "Rb" | "Cs" | "Fr" | "Be" | "Mg" | "Ca" | "Sr" | "Ba" | "Ra" | "Sc" | "Y" | "Ti" | "Zr" | "Hf" | "Db" | "V" | "Nb" | "Ta" | "Ji" | "Cr" | "Mo" | "W" | "Rf" | "Mn" | "Tc" | "Re" | "Bh" | "Fe" | "Ru" | "Os" | "Hn" | "Co" | "Rh" | "Ir" | "Mt" | "Ni" | "Pd" | "Pt" | "Cu" | "Ag" | "Au" | "Zn" | "Cd" | "Hg" | "B" | "Al" | "Ga" | "In" | "Ti" | "C" | "Si" | "Ge" | "Sn" | "Pb" | "N" | "P" | "As" | "Sb" | "Bi" | "O" | "S" | "Se" | "Te" | "Po" | "F" | "Cr" | "Br" | "I" | "At" | "He" | "Ne" | "Ar" | "Kr" | "Xe" | "Rn") & I             & Elemento                                                               \\\hline
        <VALENCIA>                     & [1-9]                                                                                                                                                                                                                                                                                                                                                                                                                                                                                                                                                        & 2             & Valor                                                                  \\\hline
        <GRUPO \_FUNCIONAL>            & ( \{GRUPO \_FUNCIONAL \_INFERIOR\} \{GRUPO \_FUNCIONAL \_SUPERIOR\} | \{GRUPO \_FUNCIONAL \_SUPERIOR\} \{GRUPO \_FUNCIONAL\_INFERIOR\} | "(" \{MODELO \_GRUPO \_FUNCIONAL\} ")" | "[" {MODELO \_GRUPO \_FUNCIONAL} "]")                                                                                                                                                                                                                                                                                                                                      & (CH3)\{Ag2\}  & Grupos funcionales, grupo funcional inferior, grupo funcional superior \\\hline
        <GRUPO \_FUNCIONAL \_INFERIOR> & "[" \{MODELO \_GRUPO \_FUNCIONAL\} "]"                                                                                                                                                                                                                                                                                                                                                                                                                                                                                                                       & [CVHe3]       & Elementos, valencias                                                   \\\hline
        <GRUPO \_FUNCIONAL \_SUPERIOR> & "(" \{MODELO \_GRUPO \_FUNCIONAL\} ")"                                                                                                                                                                                                                                                                                                                                                                                                                                                                                                                       & (CVHe3)       & Elementos, valencias                                                   \\\hline
        <MODELO \_GRUPO \_FUNCIONAL>   & (\{ELEMENTO \_QUIMICO\}+ \{VALENCIA\}?)+ | (\{ELEMENTO\}+ \{ENLACE\} \{ELEMENTO\}+)+                                                                                                                                                                                                                                                                                                                                                                                                                                                                         & FeH=C3Si4     & Elementos, enlaces, valencias                                          \\\hline
        <ENLACE>                       & ("-" | "=" | ":" | "::")                                                                                                                                                                                                                                                                                                                                                                                                                                                                                                                                     & -             & Valencia                                                               \\\hline
        \caption{Tabla de Componentes Léxicos de AVISMO}
        \label{table:lexTable}
    \end{longtable}
\end{landscape}

\section{Diseño del del Analizador Léxico}

\subsection{Autómatas Finitos Deterministas}

\begin{figure}[H]
    \footnotesize
    \begin{minipage}{.5\textwidth}
        \centering
        \begin{mdframed}
            \begin{tikzpicture}[node distance = 2.5cm, on grid, auto]
                \node [state,initial] (q0) {$0$};
                \node [state,accepting] [right=of q0] (q1) {$1$};
                \node [state,accepting] [below=of q1] (q2) {$2$};

                \path[-stealth, thick]
                (q0) edge node {;} (q1)
                (q0) edge [bend right] node {:} (q2);
            \end{tikzpicture}
            \label{fig: finDeLineaAutomata}
        \end{mdframed}
        \caption{Automata del patrón para el token <FIN\_DE\_LINEA>}
    \end{minipage}\hspace{1cm}
    \begin{minipage}{0.5\textwidth}
        \centering
        \begin{mdframed}
            \begin{tikzpicture}[node distance = 2.5cm, on grid, auto]
                \node [state, initial] (q0) {$0$};
                \node [state, accepting] [right=of q0] (q1) {$1$};
                \node [state, accepting] [below=of q1] {$1$};

                \path[-stealth, thick]
                (q0) edge node {defina} (q1)
                (q0) edge [bend right] node {como} (q2);
            \end{tikzpicture}
        \end{mdframed}
        \label{fig: palabraReservada}
        \captionof{figure}{Automata del patrón para el token <PALABRAS\_RESERVADA>}
    \end{minipage}
\end{figure}


\begin{figure}[H]
    \footnotesize
    \begin{minipage}{0.5\textwidth}
        \begin{mdframed}
            \begin{tikzpicture}[node distance = 2.5cm, auto]
                \node [state, initial] (q0) {0};
                \node [state, accepting] [right=of q0] (q1) {$1$};
                \node [state] [below=of q0] (q2) {$2$};
                \node [state, accepting] [right=of q2] (q3) {$3$};

                \path[-stealth, thick]
                (q0) edge node {[A-Za-z]} (q1)
                (q0) edge node {<LETRA>} (q2)
                (q2) edge node {<IDCONT>} (q3);
            \end{tikzpicture}
        \end{mdframed}
        \label{fig: idAutomata}
        \caption{Automata del patrón para el token <ID>}
    \end{minipage}\hspace{1cm}
    \begin{minipage}{0.5\textwidth}
        \begin{mdframed}
            \begin{tikzpicture}[node distance = 1.5cm, auto]
                \node [state, initial] (q0) {$0$};
                \node [state, accepting] [right=of q0] (q1) {$1$};
                \node [state] [below=of q1] (q2) {$2$};
                \node [state, accepting] [above left=of q0] (q4) {$4$};
                \node [state] [below left=of q0] (q5) {$5$};


                \path[-stealth, thick]
                (q0) edge node {[A-Za-z]} (q1)
                (q0) edge node [left] {<LETRA>} (q2)
                (q2) edge [bend right] node [right]{<IDCONT>} (q0)
                (q0) edge [bend right] node[right] {[0-9]} (q4)
                (q0) edge [bend right] node [left] {<DIGITO>} (q5)
                (q5.east) edge [bend right] node [below] {<IDCONT>} (q0);


            \end{tikzpicture}
        \end{mdframed}
        \label{fig: idContAutomata}
        \caption{Automata del patrón para el token <IDCONT>}
    \end{minipage}
\end{figure}

\begin{figure}[H]
    \footnotesize
    \begin{minipage}{0.5\textwidth}
        \begin{mdframed}
            \begin{tikzpicture}[node distance = 1.5cm, on grid, auto]
                \node [state,initial] (q0) {$0$};
                \node [state,accepting] [right=of q1] (q1) {$1$};

                \path[-stealth,thick]
                (q0) edge node {=} (q1);
            \end{tikzpicture}
        \end{mdframed}
        \label{fig: asigAutomata}
        \caption{Automata del patrón para el token <ASIGNACION>}
    \end{minipage}\hspace{1cm}
    \begin{minipage}{0.5\linewidth}
        \begin{mdframed}
            \begin{tikzpicture}[node distance = 0cm, on grid ,auto]
                \node [state,initial] (q0) {$0$};
                \node [state,accepting] [right=of q1] (q1) {$1$};

                \path[-stealth,thick]
                (q0) edge node {[0-9]} (q1);
            \end{tikzpicture}
        \end{mdframed}
        \label{fig: letraAutomata}
        \caption{Automata del patrón para el token <LETRA>}
    \end{minipage}
\end{figure}

\begin{center}
    \begin{figure}[H]
        \footnotesize
        \begin{mdframed}
            \begin{tikzpicture}[node distance = 0cm, on grid ,auto]
                \node [state,initial] (q0) {$0$};
                \node [state,accepting] [right=of q1] (q1) {$1$};

                \path[-stealth,thick]
                (q0) edge node {[0-9]} (q1);
            \end{tikzpicture}
        \end{mdframed}
        \label{fig: digitoAutomata}
        \caption{Automata del patrón para el token <DIGITO>}
    \end{figure}
\end{center}

\subsection{Tabla de Símbolos}

\begin{figure}[H]
    \begin{tikzpicture}[
            node distance=2cm and 3cm,
            ID/.style={rectangle, rounded corners, draw=black, thick, minimum size=10mm},
            ATT/.style={rectangle, rounded corners, draw=red!60, thick, minimum size=10mm},
        ]
        \draw node at (0,1.6)   {Identificador};
        \draw node at (7.5, 1.6)  {Atributo};

        \node[ID]   (var1)                  {var1};
        \node[ID]   (Big)   [below=of var1] {Big};

        \node[ATT]  (str1)  [right=of var1]  {str, val:"chungus", esMutable:false};
        \node[ATT]  (int1)  [right=of Big]   {int, val:"12",      esMutable:false};

        \draw[->, very thick] (var1.east) to (str1.west);
        \draw[->, very thick] (Big.east) to (int1.west);

        \node[draw=black, thick, fit={(var1) (Big)}, inner sep=10pt] (box) {};
        \node[draw=black, thick, fit={(str1) (int1)}, inner sep=10pt] (box) {};
    \end{tikzpicture}
    \label{fig: tablaDeSimbolos}
    \caption{Tabla de símbolos implementada como un diccionario}
\end{figure}

\section{Implementación del Analizador Léxico}


Como se menciona en la introducción de este proyecto, se toma la implementación lexica y sintáctica del proyecto mediante la implementación de Calc++ por bwasti 
\cite{wasti_bwastibison-example-calc-_2020}, por lo que se maneja el programa como una caja negra. Se añade edito al programa ejemplar que existia en este
repositorio (como cambiar la gramatica por la de AVISMO) pero dejandonos llevar del ejemplo original.

Es importante comenzar con que AVISMO siempre tiene que empezar ejecutando el comando \textit{MAKE CLEAN}, debido a que se tiene que crear el ejecutable del
programa. Esto entonces creara todas las dependencias necesarias dentro del programa, compilando Bison y Flex (con todos los .cc, .hh y .ll files) y 
complila todos los programas objetos (.o) que se crean para el programa driver (encargado de abrir archivos input e instanciar el analizador), parser y scanner. 
Posterior a esto es que se puede ejecutar \textit{$$./avismo fileName.txt$$} el cual para nuestro caso seria el siguiente: \textit{$$./avismo test_prog.txt$$}

Incurriendo esto, el programa entonces se encargara de leer \textbf{cada caracter} del programa e identificar si en el programa existen símbolos, identificadores, o
palabras reservadas de distintos tipos definidas en la gramatica. Se pasan por el programa intermediario \textit{driver.cc} el cual se encarga de abrir el archivo de
palabras reservadas (\textit{keywords.txt}) y entra al proceso de analisis o "parsing". Aqui entonces verifica las definiciones de la gramatica, las cuales 
determinan su tipo y los patrones de caracteres que se aceptaran y tokenizan. En nuestro caso, tenemos como ejemplo token <std::string> FIN\_DE\_LINEA "fin\_de\_linea" 
donde se especifica que es un token tipo string el cual se denota como FIN\_DE\_LINEA (determina que ya la sentencia se acabo) fin\_de\_linea es cómo se identifica 
en la expresión regular de cómo el programa se ejecuta.

El programa mismo necesita un patrón lógico para su ejecución, el cual definimos en base a la gramatica cómo:


\%\%


\%start unit;


unit: exps {};


exps: \%empty | exps exp {};


////assignment: "identificador" "="


exp:
"fin\_de\_linea" {}
| "letra" {}
| "digito" {}
| "tipo" {}
| "operacion" {}
| "valencia" {}
| "enlace" {}
| "palabra\_reservada" {}
| "id" {}
| "idcont" {}
| "sentencias" {}
| "sentencia" {}
| "modelo\_molecular" {}
| "compuesto" {}
| "compuestos" {}
| "elemento" {}
| "elemento\_quimico" {}
| "grupo\_funcional" {}
| "grupo\_funcional\_inferior" {}
| "grupo\_funcional\_superior" {}
| "modelo\_grupo\_funcional" {}\\
\%\%

\newpage
El formato del programa debe seguir este patron donde se empieza con el axioma \textit{unit}, el cual esta compuesto por \textit{exps{}} (expresiones) el cual 
se contiene a si mismo (definición recursiva) y a \textit{exp} (expresión), la cual esta compuesta por todos los tipos de tokens que podemos identificar 
con el analizador léxico. Aqui entonces introducimos el scanner, el cual contiene los patrones que debe seguir cada uno de los tokens que previamente hemos definido.
Es aqui donde definimos el formato del archivo de salida para presentar los lexemas que se tokenizaron o los errores devueltos:


std::ofstream\ file("output.txt");\\
\indent void format\_output (std::string token,const char* yytext, yy::location\& loc) \{\\
\indent file << "(" << "<" << token << ">," << std::string(yytext) << "," << loc << ")" << std::endl;\\
\indent\}

Las definiciones que utilizamos para este archivo \textit{scanner.ll} son las mismas que fueron presentadas en la sección anterior 2.1 Gramatica del Lenguaje avismo 
pero es en este archivo que damos la definición de qué cada función ejecutará una vez tenga estos lexemas que siguen el patrón establecido (tokenizados). Se utilizara 
una variable llamada loc que contiene la referencia a memoria de la variable localización de la clase driver, el cual usaremos para retornar el valor de la linea en 
donde encontramos un lexema dado. Tambien esta función devuelve tanto el nombre del token que se categoriza, y su valor como se envia al parser como yytext (contiene el valor del lexema); 
todos estos valores como arguments para especificar el formato que queremos el archivo de salida (\textit{output.txt}):


format\_output("SENTENCIAS",yytext,loc);


Todas las definiciones (excepto \textbf{ID}) contienen la función de format\_output para la salida de nuestro programa y siempre devuelven una función que se 
utilizara en el parser (que Flex y Bison se encargan en crear por nosotros) para crear una instancia del token.

{SENTENCIAS} \{\\
	\indent\indent format\_output("SENTENCIAS",yytext,loc);\\
	\indent\indent return yy::parser::make\_SENTENCIAS(yytext,loc);\\
    \indent \}

En el caso de \textbf{ID}, hay que hacer una diferenciación entre si es una palabra reservada o si es un identificador, pues si no se pone una restricción el programa 
no pudiera saber cual caso seria. Para eso se crea un archivo \textit{keywords.txt} el cual contiene estas palabras reservadas los cuales se cargan con el 
programa driver. En el caso de que el valor del lexema que esta en "variables" en la clase de drv (driver, la cual contiene el lexema que se esta leyendo) resulta ser 
el mismo valor que yytext (el cual contiene el lexema retornado del patrón reconocido) y cuyo texto se encuentra en la lista y su valor es vacio (se sabe por cómo se llena 
las palabras a driver al momento de ejecutar el constructor), sabemos entonces que cumple con todos los requisitos de ser una palabra reservada en el contexto del programa.
De no ser el caso, pues entonces no hay otra opción que ser un identificador.


{ID} \{	
  std::string text(yytext);
  if (drv.variables.find(text) != drv.variables.end() \&\& drv.variables[text] == "") {
    if (drv.variables[text] == "") {
      format\_output("PALABRA\_RESERVADA",yytext,loc);
      return yy::parser::make\_PALABRA\_RESERVADA (yytext,loc);
    }
  }
	format\_output("ID",yytext,loc);
	return yy::parser::make\_ID(text,loc);
\}