\section{Conceptos básicos}

En la tabla \ref{table:lexTable}, en la columna de patrones, note que cuando dice $\left\{\textit{TOKEN}\right\}$ donde \textit{TOKEN} se refiere a el patrón asociado a \textit{token}.
Por ejemplo, si un patrón dice $\left\{\textit{ELEMENTO\_QUIMICO}\right\}$, esto significa que inserta el patrón asociado al \textit{token} \textit{ELEMENTO\_QUIMICO}.
Esto no significa que el analizador léxico espera un \textit{token} de por si, sencillamente se hizo con el propósito de evitar redundancias.

\newpage

\section{Especificación del analizador léxico}

\begin{landscape}
    \footnotesize
    \begin{longtable}{| p{0.2\textheight} | p{0.75\textheight} | p{0.2\textheight} | p{0.25\textheight} |}
        \hline
        \textit{Token}                 & Patrón                                                                                                                                                                                                                                                                                                                                                                                                                                                                                                                                                       & Lexema        & Atributos                                                              \\\hline
        <FIN\_DE\_LINEA>               & ; | :                                                                                                                                                                                                                                                                                                                                                                                                                                                                                                                                                        & :             & Símbolo reservado                                                      \\\hline
        <PALABRA \_RESERVADA>          & defina | como                                                                                                                                                                                                                                                                                                                                                                                                                                                                                                                                                & defina        & Palabra reservada                                                      \\\hline
        <ID>                           & [A-Za-z][A-Za-z0-9]*                                                                                                                                                                                                                                                                                                                                                                                                                                                                                                                                         & var1          & Modelo molecular asociado                                              \\\hline
        <IDCONT>                       & \textit{[A-Za-z0-9]+}                                                                                                                                                                                                                                                                                                                                                                                                                                                                                                                                        & 1ar           & ID asociado                                                            \\\hline
        <LETRA>                        & [A-Za-z]                                                                                                                                                                                                                                                                                                                                                                                                                                                                                                                                                     & a             & ID asociado                                                            \\\hline
        <DIGITO>                       & [0-9]                                                                                                                                                                                                                                                                                                                                                                                                                                                                                                                                                        & 7             & Valor numérico, lexema asociado                                        \\\hline
        <TIPO>                         & modelo                                                                                                                                                                                                                                                                                                                                                                                                                                                                                                                                                       & modelo        & ID asociado                                                            \\\hline
        <OPERACION>                    & graficar2d | graficar3d | pesomolecular                                                                                                                                                                                                                                                                                                                                                                                                                                                                                                                      & pesomolecular & ID asociado                                                            \\\hline
        <MODELO \_MOLECULAR>           & (\{ELEMENTO \_QUIMICO\} | \{ELEMENTO \_QUIMICO\} \{VALENCIA\} | \{ELEMENTO\} \{GRUPO \_FUNCIONAL\} | \{ELEMENTO\} \{GRUPO \_FUNCIONAL\} \{ENLACE\} | \{ELEMENTO\} \{ENLACE\})                                                                                                                                                                                                                                                                                                                                                                                & CH3(CH3)CHH   & ID asociado                                                            \\\hline
        <COMPUESTO>                    & COMPUESTO (\{ELEMENTO \_QUIMICO\} | \{ELEMENTO \_QUIMICO\} \{VALENCIA\} | \{ELEMENTO\} \{GRUPO\_FUNCIONAL\} | \{ELEMENTO\} \{GRUPO \_FUNCIONAL\} \{ENLACE\} | \{ELEMENTO\} \{ENLACE\})                                                                                                                                                                                                                                                                                                                                                                       & CH3::         & Modelo molecular asociado, enlaces, valencias                          \\\hline
        <COMPUESTOS>                   & \{COMPUESTO\}+                                                                                                                                                                                                                                                                                                                                                                                                                                                                                                                                               & CH3::(OH)3    & Modelo molecular asociado, enlaces, valencias                          \\\hline
        <ELEMENTO>                     & \{ELEMENTO \_QUIMICO\} \{VALENCIA\}?                                                                                                                                                                                                                                                                                                                                                                                                                                                                                                                         & Ag3           & Elemento, valencia                                                     \\\hline
        <ELEMENTO \_QUIMICO>           & ( "H" | "Li" | "Na" | "K" | "Rb" | "Cs" | "Fr" | "Be" | "Mg" | "Ca" | "Sr" | "Ba" | "Ra" | "Sc" | "Y" | "Ti" | "Zr" | "Hf" | "Db" | "V" | "Nb" | "Ta" | "Ji" | "Cr" | "Mo" | "W" | "Rf" | "Mn" | "Tc" | "Re" | "Bh" | "Fe" | "Ru" | "Os" | "Hn" | "Co" | "Rh" | "Ir" | "Mt" | "Ni" | "Pd" | "Pt" | "Cu" | "Ag" | "Au" | "Zn" | "Cd" | "Hg" | "B" | "Al" | "Ga" | "In" | "Ti" | "C" | "Si" | "Ge" | "Sn" | "Pb" | "N" | "P" | "As" | "Sb" | "Bi" | "O" | "S" | "Se" | "Te" | "Po" | "F" | "Cr" | "Br" | "I" | "At" | "He" | "Ne" | "Ar" | "Kr" | "Xe" | "Rn") & I             & Elemento                                                               \\\hline
        <VALENCIA>                     & [1-9]                                                                                                                                                                                                                                                                                                                                                                                                                                                                                                                                                        & 2             & Valor                                                                  \\\hline
        <GRUPO \_FUNCIONAL>            & ( \{GRUPO \_FUNCIONAL \_INFERIOR\} \{GRUPO \_FUNCIONAL \_SUPERIOR\} | \{GRUPO \_FUNCIONAL \_SUPERIOR\} \{GRUPO \_FUNCIONAL\_INFERIOR\} | "(" \{MODELO \_GRUPO \_FUNCIONAL\} ")" | "[" {MODELO \_GRUPO \_FUNCIONAL} "]")                                                                                                                                                                                                                                                                                                                                      & (CH3)\{Ag2\}  & Grupos funcionales, grupo funcional inferior, grupo funcional superior \\\hline
        <GRUPO \_FUNCIONAL \_INFERIOR> & "[" \{MODELO \_GRUPO \_FUNCIONAL\} "]"                                                                                                                                                                                                                                                                                                                                                                                                                                                                                                                       & [CVHe3]       & Elementos, valencias                                                   \\\hline
        <GRUPO \_FUNCIONAL \_SUPERIOR> & "(" \{MODELO \_GRUPO \_FUNCIONAL\} ")"                                                                                                                                                                                                                                                                                                                                                                                                                                                                                                                       & (CVHe3)       & Elementos, valencias                                                   \\\hline
        <MODELO \_GRUPO \_FUNCIONAL>   & (\{ELEMENTO \_QUIMICO\}+ \{VALENCIA\}?)+ | (\{ELEMENTO\}+ \{ENLACE\} \{ELEMENTO\}+)+                                                                                                                                                                                                                                                                                                                                                                                                                                                                         & FeH=C3Si4     & Elementos, enlaces, valencias                                          \\\hline
        <ENLACE>                       & ("-" | "=" | ":" | "::")                                                                                                                                                                                                                                                                                                                                                                                                                                                                                                                                     & -             & Valencia                                                               \\\hline
        \caption{Tabla de Componentes Léxicos de AVISMO}
        \label{table:lexTable}
    \end{longtable}
\end{landscape}

\section{Diseño del Analizador Léxico}

\subsection{Autómatas Finitos Deterministas}

\begin{figure}[H]
    \footnotesize
    \begin{minipage}{.5\textwidth}
        \centering
        \begin{mdframed}
            \begin{tikzpicture}[node distance = 2.5cm, on grid, auto]
                \node [state,initial] (q0) {$0$};
                \node [state,accepting] [right=of q0] (q1) {$1$};
                \node [state,accepting] [below=of q1] (q2) {$2$};

                \path[-stealth, thick]
                (q0) edge node {;} (q1)
                (q0) edge [bend right] node {:} (q2);
            \end{tikzpicture}
            \label{fig: finDeLineaAutomata}
        \end{mdframed}
        \caption{Autómata del patrón para el token <FIN\_DE\_LINEA>}
    \end{minipage}\hspace{1cm}
    \begin{minipage}{0.5\textwidth}
        \centering
        \begin{mdframed}
            \begin{tikzpicture}[node distance = 2.5cm, on grid, auto]
                \node [state, initial] (q0) {$0$};
                \node [state, accepting] [right=of q0] (q1) {$1$};
                \node [state, accepting] [below=of q1] {$1$};

                \path[-stealth, thick]
                (q0) edge node {defina} (q1)
                (q0) edge [bend right] node {como} (q2);
            \end{tikzpicture}
        \end{mdframed}
        \label{fig: palabraReservada}
        \captionof{figure}{Autómata del patrón para el token <PALABRAS\_RESERVADA>}
    \end{minipage}
\end{figure}


\begin{figure}[H]
    \footnotesize
    \begin{minipage}{0.5\textwidth}
        \begin{mdframed}
            \begin{tikzpicture}[node distance = 2.5cm, auto]
                \node [state, initial] (q0) {0};
                \node [state, accepting] [right=of q0] (q1) {$1$};
                \node [state] [below=of q0] (q2) {$2$};
                \node [state, accepting] [right=of q2] (q3) {$3$};

                \path[-stealth, thick]
                (q0) edge node {[A-Za-z]} (q1)
                (q0) edge node {<LETRA>} (q2)
                (q2) edge node {<IDCONT>} (q3);
            \end{tikzpicture}
        \end{mdframed}
        \label{fig: idAutomata}
        \caption{Autómata del patrón para el token <ID>}
    \end{minipage}\hspace{1cm}
    \begin{minipage}{0.5\textwidth}
        \begin{mdframed}
            \begin{tikzpicture}[node distance = 1.5cm, auto]
                \node [state, initial] (q0) {$0$};
                \node [state, accepting] [right=of q0] (q1) {$1$};
                \node [state] [below=of q1] (q2) {$2$};
                \node [state, accepting] [above left=of q0] (q4) {$4$};
                \node [state] [below left=of q0] (q5) {$5$};


                \path[-stealth, thick]
                (q0) edge node {[A-Za-z]} (q1)
                (q0) edge node [left] {<LETRA>} (q2)
                (q2) edge [bend right] node [right]{<IDCONT>} (q0)
                (q0) edge [bend right] node[right] {[0-9]} (q4)
                (q0) edge [bend right] node [left] {<DIGITO>} (q5)
                (q5.east) edge [bend right] node [below] {<IDCONT>} (q0);


            \end{tikzpicture}
        \end{mdframed}
        \label{fig: idContAutomata}
        \caption{Autómata del patrón para el token <IDCONT>}
    \end{minipage}
\end{figure}

\begin{figure}[H]
    \footnotesize
    \begin{minipage}{0.5\textwidth}
        \begin{mdframed}
            \begin{tikzpicture}[node distance = 1.5cm, on grid, auto]
                \node [state,initial] (q0) {$0$};
                \node [state,accepting] [right=of q1] (q1) {$1$};

                \path[-stealth,thick]
                (q0) edge node {=} (q1);
            \end{tikzpicture}
        \end{mdframed}
        \label{fig: asigAutomata}
        \caption{Autómata del patrón para el token <ASIGNACION>}
    \end{minipage}\hspace{1cm}
    \begin{minipage}{0.5\linewidth}
        \begin{mdframed}
            \begin{tikzpicture}[node distance = 0cm, on grid ,auto]
                \node [state,initial] (q0) {$0$};
                \node [state,accepting] [right=of q1] (q1) {$1$};

                \path[-stealth,thick]
                (q0) edge node {[0-9]} (q1);
            \end{tikzpicture}
        \end{mdframed}
        \label{fig: letraAutomata}
        \caption{Autómata del patrón para el token <LETRA>}
    \end{minipage}
\end{figure}

\begin{center}
    \begin{figure}[H]
        \footnotesize
        \begin{mdframed}
            \begin{tikzpicture}[node distance = 0cm, on grid ,auto]
                \node [state,initial] (q0) {$0$};
                \node [state,accepting] [right=of q1] (q1) {$1$};

                \path[-stealth,thick]
                (q0) edge node {[0-9]} (q1);
            \end{tikzpicture}
        \end{mdframed}
        \label{fig: digitoAutomata}
        \caption{Autómata del patrón para el token <DIGITO>}
    \end{figure}
\end{center}

\subsection{Tabla de Símbolos}

\begin{figure}[H]
    \begin{tikzpicture}[
            node distance=2cm and 3cm,
            ID/.style={rectangle, rounded corners, draw=black, thick, minimum size=10mm},
            ATT/.style={rectangle, rounded corners, draw=red!60, thick, minimum size=10mm},
        ]
        \draw node at (0,1.6)   {Identificador};
        \draw node at (7.5, 1.6)  {Atributo};

        \node[ID]   (var1)                  {var1};
        \node[ID]   (Big)   [below=of var1] {Big};

        \node[ATT]  (str1)  [right=of var1]  {str, val:"chungus", esMutable:false};
        \node[ATT]  (int1)  [right=of Big]   {int, val:"12",      esMutable:false};

        \draw[->, very thick] (var1.east) to (str1.west);
        \draw[->, very thick] (Big.east) to (int1.west);

        \node[draw=black, thick, fit={(var1) (Big)}, inner sep=10pt] (box) {};
        \node[draw=black, thick, fit={(str1) (int1)}, inner sep=10pt] (box) {};
    \end{tikzpicture}
    \label{fig: tablaDeSimbolos}
    \caption{Tabla de símbolos implementada como un diccionario}
\end{figure}

\section{Implementación del Analizador Léxico}


Como se ha mencionado anteriormente, la implementación lexica del proyecto fue inspirada por la MAPL \cite{noauthor_pl-project-lgm-yvv-amnmapl_nodate} y adaptada para la gramática de AVISMO.

Para ejecutar el analizador léxico, primero se tiene que instalar \textit{Python 3}. Luego, se ejecuta \mintinline{Bash}|pip install ply| por la línea de comando.
Finalmente, se ejecuta \mintinline{Bash}|python lexer.py test_prog.txt|, donde \mintinline{Bash}|test_prog| es el archivo de entrada escrito en AVISMO.


Al ejecutarse el comando anterior, el programa procede a leer \textbf{cada caracter} del programa e identificar si una serie de caracteres sigue un patrón que forma parte del lenguaje AVISMO. Al encontrar un patrón reconocido, tales como un identificador o modelo molecular, lo clasifica con un \textit{token} correspondiente, lo emprime en el archivo de \href{https://github.com/aramis-matos/comp4999_compilers_project/blob/master/code/python_remake/output.txt}{\textit{output.txt}} y lo devuelve al analizador sintáctico. Note, que el patrón de identificador reconoce palabra reservadas también. Esto crea ambigüedad semántica debido a que la gramatica no tiene un mecanismo para diferenciar entre una palabra reservada y un identificador. Por esta razón, si una serie de caracteres se identifica como un lexema de categoría identificador, se compara con los valores ya existentes del diccionario \textit{variables}. Al inicializar el programa \mintinline{Bash}|lexer.py|, este se encarga de abrir el archivo de
palabras reservadas (\href{https://github.com/aramis-matos/comp4999_compilers_project/blob/master/code/python remake/keywords.txt}{\textit{keywords.txt}}) y añadir las palabras reservadas antes que cualquier variable se pueda inicializar. Más aún, a las palabras reservadas se les asigna el valor de la cadena vacía. Esto se hace con el propósito de poder diferenciar entre palabras reservadas e identificadores, ya que al nivel sintáctico, no es posible asignarle a una identificador una cadena vacía, como se puede apreciar a continuación:
\begin{figure}[H]
\begin{minted}[breaklines,breakanywhere]{python}
def t_ID(t):                #identifica el token ID pero tambien idetifica el token de palabra reservada
    r"[A-Za-z]+\d*"
    isPR = reserved.get(t.value,"ID") 
    if isPR != "ID":
        t.type = "PALABRA_RESERVADA"
    else:
        t.type = isPR
        variables[t.value] = ""
    return t
\end{minted}
\caption{Patrón que cezga identificadores de palabras reservadas}
\label{fig: id}
\end{figure}
Con el propósito de visualizar los lexemas generados por \mintinline{Bash}|lexer.py|, colocado en el archivo \href{https://github.com/aramis-matos/comp4999_compilers_project/blob/master/code/python_remake/lexer.py}{\textit{lexer.py}}, se utiliza la siguiente función:
\begin{figure}[H]
    \begin{minted}[breaklines,breakanywhere,linenos]{python}
with open(test_file, "r") as f:
    with open("output.txt","w") as o:
        for data in f:
            lexer.input(data)
            for tok in lexer:
                tokenTable.add_row([tokenNum,tok.type,tok.value,tok.lineno,tok.lexpos,test_file])
                tokenNum += 1
        o.write(str(tokenTable))
        line = "\n\nTABLA DE SIMBOLOS"
        o.write(line+"\n")
        for val in variables:
            symbofigable.add_row([val])
        o.write(str(symbofigable)+"\n")
        line = "\n\nPALABRAS RESERVADAS"
        o.write(line+"\n")
        for val in reserved:
            reservedWords.add_row([val])
        o.write(str(reservedWords)+"\n")
\end{minted}
    \caption{Código para imprimir los \textit{tokens} encontrados}
    \label{fig: scanner}
\end{figure}
Esto imprime el \textit{token} resultante en el archivo \href{https://github.com/aramis-matos/comp4999_compilers_project/blob/master/code/python_remake/output.txt}{\textit{output.txt}}. En el caso de un error léxico, se ejecuta el siguiente código:
\begin{figure}[H]
\begin{minted}[breaklines,breakanywhere]{python}
def t_error(t):             # identifica error lexico
    tokenTable.add_row([tokenNum,"ERROR",t.value[0],t.lineno,t.lexpos,test_file])
    t.lexer.skip(1)
\end{minted}
\caption{Gestión de errores}
\label{fig: errores}
\end{figure}
Cuando se encuentra un error léxico, no se retorna al analizador sintáctico. Los patrones que se utilizan están en \mintinline{Bash}|lexer.py| (líneas 58-103) son una adaptación de la gramática en la tabla \ref{table:lexTable}. Note la tupla \textit{tokens}.
Esta contiene todos los \textit{tokens} del lenguaje AVISMO. Sin embargo, no todos están definidos a el nivel léxico. La mayoría de los \textit{tokens} se construyen en la etapa sintáctica, particularmente los que tienen que ver con compuestos químicos y modelos moleculares.
Más aún, note el objeto \textit{tok} en la línea 5 de el figura \ref{fig: scanner}.
Este contiene los atributos \textit{type}, \textit{value}, \textit{lineno} y \textit{lexpos}.
Estos devuelven, respectivamente, el tipo, lexema, en que línea del archivo se encuentra y la posición del primer caracter de el \textit{token} llamado \textit{tok}. 
Otro detalle importante de la implementación léxica es el orden de aplicación de los patrones. 
Los patrones en forma de variables, como en el figura \ref{fig: patronVariable}, tienen que estar escrito antes que los patrones escritos en forma de funciones, como en el figura \ref{fig: id} y \ref{fig: patronFuncion}.

\begin{figure}[H]
\begin{minted}[breaklines,breakanywhere]{python}
    t_ENLACE = r"(-|=|:|::)" #define los tokens para diferentes tipos de enlaces quimicos
\end{minted}
    \caption{Patrón en forma de variable}
    \label{fig: patronVariable}
\end{figure}

\begin{figure}[H]
    \begin{minted}[breaklines,breakanywhere]{python}
def t_ELEMENTO_QUIMICO(t):  #define regla para el token elemento quimico 
    r"(H|Li|Na|K|Rb|Cs|Fr|Be|Mg|Ca|Sr|Ba|Ra|Sc|Y|Ti|Zr|Hf|Db|V|Nb|Ta|Ji|Cr|Mo|W|Rf|Mn|Tc|Re|Bh|Fe|Ru|Os|Hn|Co|Rh|Ir|Mt|Ni|Pd|Pt|Cu|Ag|Au|Zn|Cd|Hg|B|Al|Ga|In|Ti|C|Si|Ge|Sn|Pb|N|P|As|Sb|Bi|O|S|Se|Te|Po|F|Cr|Br|I|At|He|Ne|Ar|Kr|Xe|Rn)"
    return t
\end{minted}
    \caption{Patrón en forma de función}
    \label{fig: patronFuncion}
\end{figure}

\newpage

\section{Casos de prueba para el analizador léxico}
\label{sec:lexerTest}

\subsection{Tres programas léxicamente correctos}
\label{subsec: lexerCorr}
\begin{figure}[H]
    \scaledimage{images/lexcorr1.png}
    \caption{Ejemplo de Programa Correcto}
    \label{fig: lexcorr1}
\end{figure}

\begin{figure}[H]
    \scaledimage{images/lexcorr2.png}
    \caption{Ejemplo de Programa Correcto}
    \label{fig: lexcorr2}
\end{figure}

\begin{figure}[H]
    \scaledimage{images/lexcorr3.png}
    \caption{Ejemplo de Programa Correcto}
    \label{fig: lexcorr3}
\end{figure}

\newpage
\subsection{Tres programas léxicamente incorrectos}

\label{subsec: lexerIncorr}
\begin{figure}[H]
    \scaledimage{images/lexincorr1.png}
    \caption{Ejemplo de Programa Incorrecto}
    \label{fig: lexincorr1}
\end{figure}

\begin{figure}[H]
    \scaledimage{images/lexincorr2.png}
    \caption{Ejemplo de Programa Incorrecto}
    \label{fig: lexincorr2}
\end{figure}

\begin{figure}[H]
    \scaledimage{images/lexincorr3.png}
    \caption{Ejemplo de Programa Incorrecto}
    \label{fig: lexincorr3}
\end{figure}
