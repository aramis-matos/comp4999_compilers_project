\section{Especificación del Analizador Léxico}

\begin{table}[ht]
    \footnotesize
    \begin{tabularx}{\linewidth}{|X|X|X|X|}
        \hline
        Lexema & \textit{Token}        & Patrones                                                              & Atributo                                          \\\hline
        ;      & <FIN\_DE\_ LINEA>     & ; | :                                                                 & Indica fin de Línea                               \\\hline
        defina & <PALABRA\_ RESERVADA> & defina | como                                                         & Indica declaración de una variable                \\\hline
        VaR1   & <ID>                  & [A-Za-z] | <LETRA> <IDCONT>                                           & Apuntador a la tabla de símbolos                  \\\hline
        X1     & <IDCONT>              & \textit{[A-Za-z]} | LETRA IDCONT | \textit{[0-9]} | <DIGITO> <IDCONT> & Permite que los identificadores contengan números \\\hline
        =      & <ASIGNACION>          & =                                                                     & Asigna un <MODELO\_MOLECULAR a un identificador   \\\hline
        X1y2   & <ID>                  & [A-Za-z] | <LETRA> <IDCONT>                                           & Indica declaración de una variable                \\\hline
        A      & <LETRA>               & [A-Za-z]                                                              & Provee un terminal para <ID> y <IDCONT>           \\\hline
    \end{tabularx}
    \label{table: lexTable}
    \caption{Definición Léxica del Lenguaje AVISMO}
\end{table}

\section{Diseño del del Analizador Léxico}

\begin{center}
    \begin{tikzpicture}[node distance = 4cm, on grid, auto]
        \node [state,initial] (q0) {$q_0$};
        \node [state,accepting] [right=of q0] (q1) {$q_1$};

        \path[-stealth, thick]
        (q0) edge node {1} (q1);
    \end{tikzpicture}
\end{center}


\section{Implementación del Analizador Léxico}