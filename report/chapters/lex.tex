\section{Especificación del Analizador Léxico}

\begin{table}[ht]
    \footnotesize
    \begin{tabularx}{\linewidth}{|X|X|X|X|}
        \hline
        Lexema & \textit{Token}        & Patrones                                                              & Atributo                                          \\\hline
        ;      & <FIN\_DE\_ LINEA>     & ; | :                                                                 & Indica fin de Línea                               \\\hline
        defina & <PALABRA\_ RESERVADA> & defina | como                                                         & Indica declaración de una variable                \\\hline
        VaR1   & <ID>                  & [A-Za-z] | <LETRA> <IDCONT>                                           & Apuntador a la tabla de símbolos                  \\\hline
        X1     & <IDCONT>              & \textit{[A-Za-z]} | <LETRA> <IDCONT> | \textit{[0-9]} | <DIGITO> <IDCONT> & Permite que los identificadores contengan números \\\hline
        =      & <ASIGNACION>          & =                                                                     & Asigna un <MODELO\_MOLECULAR a un identificador   \\\hline
        X1y2   & <ID>                  & [A-Za-z] | <LETRA> <IDCONT>                                           & Indica declaración de una variable                \\\hline
        A      & <LETRA>               & [A-Za-z]                                                              & Provee un terminal para <ID> y <IDCONT>           \\\hline
    \end{tabularx}
    \label{table: lexTable}
    \caption{Definición Léxica del Lenguaje AVISMO}
\end{table}

\section{Diseño del del Analizador Léxico}


\begin{figure}[ht]
    \footnotesize
    \begin{minipage}{.5\textwidth}
        \centering
        \begin{mdframed}
            \begin{tikzpicture}[node distance = 2.5cm, on grid, auto]
                \node [state,initial] (q0) {$0$};
                \node [state,accepting] [right=of q0] (q1) {$1$};
                \node [state,accepting] [below=of q1] (q2) {$2$};

                \path[-stealth, thick]
                (q0) edge node {;} (q1)
                (q0) edge [bend right] node {:} (q2);
            \end{tikzpicture}
            \label{fig: finDeLineaAutomata}
        \end{mdframed}
        \caption{Automata del patrón para el token <FIN\_DE\_LINEA>}
    \end{minipage}\hspace{1cm}
    \begin{minipage}{0.5\textwidth}
        \centering
        \begin{mdframed}
            \begin{tikzpicture}[node distance = 2.5cm, on grid, auto]
                \node [state, initial] (q0) {$0$};
                \node [state, accepting] [right=of q0] (q1) {$1$};
                \node [state, accepting] [below=of q1] {$1$};

                \path[-stealth, thick]
                (q0) edge node {defina} (q1)
                (q0) edge [bend right] node {como} (q2);
            \end{tikzpicture}
        \end{mdframed}
        \label{fig: palabraReservada}
        \captionof{figure}{Automata del patrón para el token <PALABRAS\_RESERVADA>}
    \end{minipage}
\end{figure}


\begin{figure}[ht]
    \footnotesize
    \begin{minipage}{0.5\textwidth}
        \begin{mdframed}
            \begin{tikzpicture}[node distance = 2.5cm, auto]
                \node [state, initial] (q0) {0};
                \node [state, accepting] [right=of q0] (q1) {$1$};
                \node [state] [below=of q0] (q2) {$2$};
                \node [state, accepting] [right=of q2] (q3) {$3$};

                \path[-stealth, thick]
                (q0) edge node {[A-Za-z]} (q1)
                (q0) edge node {<LETRA>} (q2)
                (q2) edge node {<IDCONT>} (q3);
            \end{tikzpicture}
        \end{mdframed}
        \label{fig: idAutomata}
        \caption{Automata del patrón para el token <ID>}
    \end{minipage}\hspace{1cm}
    \begin{minipage}{0.5\textwidth}
        \begin{mdframed}
            \begin{tikzpicture}[node distance = 1.5cm, auto]
                \node [state, initial] (q0) {$0$};
                \node [state, accepting] [right=of q0] (q1) {$1$};
                \node [state] [below=of q1] (q2) {$2$};
                \node [state, accepting] [below=of q2] (q3) {$3$};
                \node [state, accepting] [above left=of q0] (q4) {$4$};
                \node [state] [below left=of q0] (q5) {$5$};
                

                \path[-stealth, thick]
                (q0) edge node {[A-Za-z]} (q1)
                (q0) edge node {<LETRA>} (q2)
                (q2) edge node [left] {<IDCONT>} (q3)
                (q0) edge [bend right] node[right] {[0-9]} (q4)
                (q0) edge [bend right] node [left] {<DIGITO>} (q5)
                (q5) edge [bend right] node [below right] {<IDCONT>} (q0);

                
            \end{tikzpicture}
        \end{mdframed}
        \label{fig: idContAutomata}
        \caption{Automata del patrón para el token <IDCONT>}
    \end{minipage}
\end{figure}

\begin{figure}[ht]
    \footnotesize
    \begin{minipage}{0.5\textwidth}
        \begin{mdframed}
            \begin{tikzpicture}[node distance = 1.5cm, on grid, auto]
                \node [state,initial] (q0) {$0$};
                \node [state,accepting] [right=of q1] (q1) {$1$};

                \path[-stealth,thick]
                (q0) edge node {=} (q1);
            \end{tikzpicture}
        \end{mdframed}
        \label{fig: asigAutomata}
        \caption{Automata del patrón para el token <ASIGNACION>}
    \end{minipage}\hspace{1cm}
    \begin{minipage}{0.5\linewidth}
        \begin{mdframed}
            \begin{tikzpicture}[node distance = 0cm, on grid ,auto]
                \node [state,initial] (q0) {$0$};
                \node [state,accepting] [right=of q1] (q1) {$1$};

                \path[-stealth,thick]
                (q0) edge node {[0-9]} (q1);
            \end{tikzpicture}
        \end{mdframed}
        \label{fig: letraAutomata}
        \caption{Automata del patrón para el token <LETRA>}
    \end{minipage}
\end{figure}

\section{Implementación del Analizador Léxico}