\section{Gramatica del Lenguaje AVISMO}

\begin{itemize}
    \item <SENTENCIAS> ::= <FIN\_DE\_LINEA> <SENTENCIAS> | <SENTENCIA> <FIN\_DE\_LINEA>
    \item <FIN\_DE\_LINEA> ::= ":" | ";"
    \item <SENTENCIA> ::= "defina" <ID> "como" <TIPO> | <ID> "="  <MODELO\_MOLECULAR> | <OPERACION> "(" <ID> ")"
    \item <ID> ::= "A" | "B" | "C" | "D" | "E" | "F" | "G" | "H" | "I" | "J" | "K" | "L" | "M" | "N" | "O" | "P" | "Q" | "R" | "S" | "T" | "U" | "V" | "W" | "X" | "Y" | "Z" | "a" | "b" | "c" | "d" | "e" | "f" | "g" | "h" | "i" | "j" | "k" | "l" | "m" | "n" | "o" | "p" | "q" | "r" | "s" | "t" | "u" | "v" | "w" | "x" | "y" | "z" | <LETRA> <IDCONT>
    \item <IDCONT> ::= "A" | "B" | "C" | "D" | "E" | "F" | "G" | "H" | "I" | "J" | "K" | "L" | "M" | "N" | "O" | "P" | "Q" | "R" | "S" | "T" | "U" | "V" | "W" | "X" | "Y" | "Z" | "a" | "b" | "c" | "d" | "e" | "f" | "g" | "h" | "i" | "j" | "k" | "l" | "m" | "n" | "o" | "p" | "q" | "r" | "s" | "t" | "u" | "v" | "w" | "x" | "y" | "z" | <LETRA> <IDCONT> | "0" | "1" | "2" | "3" | "4" | "5" | "6" | "7" | "8" | "9" | <DIGITO> <IDCONT>
    \item <LETRA> ::= "A" | "B" | "C" | "D" | "E" | "F" | "G" | "H" | "I" | "J" | "K" | "L" | "M" | "N" | "O" | "P" | "Q" | "R" | "S" | "T" | "U" | "V" | "W" | "X" | "Y" | "Z" | "a" | "b" | "c" | "d" | "e" | "f" | "g" | "h" | "i" | "j" | "k" | "l" | "m" | "n" | "o" | "p" | "q" | "r" | "s" | "t" | "u" | "v" | "w" | "x" | "y" | "z"
    \item <DIGITO> ::= "0" | "1" | "2" | "3" | "4" | "5" | "6" | "7" | "8" | "9"
    \item <TIPO> ::= "modelo"
    \item <OPERACION> ::= "graficar2d" |"graficar3d" | "pesomolecular"
    \item <MODELO\_MOLECULAR> ::= "H" | "Li" | "Na" | "K" | "Rb" | "Cs" | "Fr" | "Be" | "Mg" | "Ca" | "Sr" | "Ba" | "Ra" | "Sc" | "Y" | "Ti" | "Zr" | "Hf" | "Db" | "V" | "Nb" | "Ta" | "Ji" | "Cr" | "Mo" | "W" | "Rf" | "Mn" | "Tc" | "Re" | "Bh" | "Fe" | "Ru" | "Os" | "Hn" | "Co" | "Rh" | "Ir" | "Mt" | "Ni" | "Pd" | "Pt" | "Cu" | "Ag" | "Au" | "Zn" | "Cd" | "Hg" | "B" | "Al" | "Ga" | "In" | "Ti" | "C" | "Si" | "Ge" | "Sn" | "Pb" | "N" | "P" | "As" | "Sb" | "Bi" | "O" | "S" | "Se" | "Te" | "Po" | "F" | "Cr" | "Br" | "I" | "At" | "He" | "Ne" | "Ar" | "Kr" | "Xe" | "Rn" | <ELEMENTO\_QUIMICO> <VALENCIA> | <ELEMENTO> <GRUPO\_FUNCIONAL> | <COMPUESTO> <ELEMENTO> <GRUPO\_FUNCIONAL> | <COMPUESTO> <COMPUESTO>
    \item <COMPUESTO> ::= "H" | "Li" | "Na" | "K" | "Rb" | "Cs" | "Fr" | "Be" | "Mg" | "Ca" | "Sr" | "Ba" | "Ra" | "Sc" | "Y" | "Ti" | "Zr" | "Hf" | "Db" | "V" | "Nb" | "Ta" | "Ji" | "Cr" | "Mo" | "W" | "Rf" | "Mn" | "Tc" | "Re" | "Bh" | "Fe" | "Ru" | "Os" | "Hn" | "Co" | "Rh" | "Ir" | "Mt" | "Ni" | "Pd" | "Pt" | "Cu" | "Ag" | "Au" | "Zn" | "Cd" | "Hg" | "B" | "Al" | "Ga" | "In" | "Ti" | "C" | "Si" | "Ge" | "Sn" | "Pb" | "N" | "P" | "As" | "Sb" | "Bi" | "O" | "S" | "Se" | "Te" | "Po" | "F" | "Cr" | "Br" | "I" | "At" | "He" | "Ne" | "Ar" | "Kr" | "Xe" | "Rn" | <ELEMENTO\_QUIMICO> <VALENCIA> | <ELEMENTO> <GRUPO\_FUNCIONAL> | <ELEMENTO> <GRUPO\_FUNCIONAL> <ENLACE> | <ELEMENTO> <ENLACE>
    \item <COMPUESTOS> ::= <COMPUESTO> <COMPUESTO> | <COMPUESTOS>
    \item <ELEMENTO> ::= "H" | "Li" | "Na" | "K" | "Rb" | "Cs" | "Fr" | "Be" | "Mg" | "Ca" | "Sr" | "Ba" | "Ra" | "Sc" | "Y" | "Ti" | "Zr" | "Hf" | "Db" | "V" | "Nb" | "Ta" | "Ji" | "Cr" | "Mo" | "W" | "Rf" | "Mn" | "Tc" | "Re" | "Bh" | "Fe" | "Ru" | "Os" | "Hn" | "Co" | "Rh" | "Ir" | "Mt" | "Ni" | "Pd" | "Pt" | "Cu" | "Ag" | "Au" | "Zn" | "Cd" | "Hg" | "B" | "Al" | "Ga" | "In" | "Ti" | "C" | "Si" | "Ge" | "Sn" | "Pb" | "N" | "P" | "As" | "Sb" | "Bi" | "O" | "S" | "Se" | "Te" | "Po" | "F" | "Cr" | "Br" | "I" | "At" | "He" | "Ne" | "Ar" | "Kr" | "Xe" | "Rn" | <ELEMENTO\_QUIMICO> <VALENCIA>
    \item <ELEMENTO\_QUIMICO> ::= "H" | "Li" | "Na" | "K" | "Rb" | "Cs" | "Fr" | "Be" | "Mg" | "Ca" | "Sr" | "Ba" | "Ra" | "Sc" | "Y" | "Ti" | "Zr" | "Hf" | "Db" | "V" | "Nb" | "Ta" | "Ji" | "Cr" | "Mo" | "W" | "Rf" | "Mn" | "Tc" | "Re" | "Bh" | "Fe" | "Ru" | "Os" | "Hn" | "Co" | "Rh" | "Ir" | "Mt" | "Ni" | "Pd" | "Pt" | "Cu" | "Ag" | "Au" | "Zn" | "Cd" | "Hg" | "B" | "Al" | "Ga" | "In" | "Ti" | "C" | "Si" | "Ge" | "Sn" | "Pb" | "N" | "P" | "As" | "Sb" | "Bi" | "O" | "S" | "Se" | "Te" | "Po" | "F" | "Cr" | "Br" | "I" | "At" | "He" | "Ne" | "Ar" | "Kr" | "Xe" | "Rn"
    \item <VALENCIA> ::= "1" | "2" | "3" | "4" | "5" | "6" | "7" | "8" | "9"
    \item <GRUPO\_FUNCIONAL> ::=
          <GRUPO\_FUNCIONAL\_INFERIOR>

          <GRUPO\_FUNCIONAL\_SUPERIOR> | <GRUPO\_FUNCIONAL\_SUPERIOR> <GRUPO\_FUNCIONAL\_INFERIOR> | "(" <MODELO\_GRUPO\_FUNCIONAL> ")" | "["  <MODELO\_GRUPO\_FUNCIONAL> "]"
    \item <GRUPO\_FUNCIONAL\_SUPERIOR> ::= "[" <MODELO\_GRUPO\_FUNCIONAL> "]"
    \item <GRUPO\_FUNCIONAL\_INFERIOR> ::= "(" <MODELO\_GRUYPO\_FUNCIONAL> ")
    \item  <MODELO\_GRUYPO\_FUNCIONAL> ::= <ENLACE> <MODELO\_MOLECULAR> | "H" | "Li" | "Na" | "K" | "Rb" | "Cs" | "Fr" | "Be" | "Mg" | "Ca" | "Sr" | "Ba" | "Ra" | "Sc" | "Y" | "Ti" | "Zr" | "Hf" | "Db" | "V" | "Nb" | "Ta" | "Ji" | "Cr" | "Mo" | "W" | "Rf" | "Mn" | "Tc" | "Re" | "Bh" | "Fe" | "Ru" | "Os" | "Hn" | "Co" | "Rh" | "Ir" | "Mt" | "Ni" | "Pd" | "Pt" | "Cu" | "Ag" | "Au" | "Zn" | "Cd" | "Hg" | "B" | "Al" | "Ga" | "In" | "Ti" | "C" | "Si" | "Ge" | "Sn" | "Pb" | "N" | "P" | "As" | "Sb" | "Bi" | "O" | "S" | "Se" | "Te" | "Po" | "F" | "Cr" | "Br" | "I" | "At" | "He" | "Ne" | "Ar" | "Kr" | "Xe" | "Rn" | <ELEMENTO\_QUIMICO> <VALENCIA> | <ELEMENTO> <GRUPO\_FUNCIONAL> | <COMPUESTO> <ELEMENTO> | <COMPUESTO> <COMPUESTO> <COMPUESTOS>
\end{itemize}
\vspace{1cm}
En la tabla \ref{table:lexTable}, en la columna de patrones, note que cuando dice $\left\{\textit{TOKEN}\right\}$ donde \textit{TOKEN} se refiere a el patrón asociado a \textit{token}.
Por ejemplo, si un patrón dice $\left\{\textit{ELEMENTO\_QUIMICO}\right\}$, esto significa que inserta el patrón asociado al \textit{token} \textit{ELEMENTO\_QUIMICO}.
Esto no significa que el analizador léxico espera un \textit{token} de por si, sencillamente se hizo con el propósito de evitar redundancias.
\newpage

\begin{landscape}
    \footnotesize
    \begin{longtable}{| p{0.2\textheight} | p{0.75\textheight} | p{0.2\textheight} | p{0.25\textheight} |}
        \hline
        \textit{Token}                 & Patrón                                                                                                                                                                                                                                                                                                                                                                                                                                                                                                                                                       & Lexema        & Atributos                                                              \\\hline
        <FIN\_DE\_LINEA>               & ; | :                                                                                                                                                                                                                                                                                                                                                                                                                                                                                                                                                        & :             & Simbolo reservado                                                      \\\hline
        <PALABRA \_RESERVADA>          & defina | como                                                                                                                                                                                                                                                                                                                                                                                                                                                                                                                                                & defina        & Palabra reservada                                                      \\\hline
        <ID>                           & [A-Za-z][A-Za-z0-9]*                                                                                                                                                                                                                                                                                                                                                                                                                                                                                                                                         & var1          & Modelo molecular asociado                                              \\\hline
        <IDCONT>                       & \textit{[A-Za-z0-9]+}                                                                                                                                                                                                                                                                                                                                                                                                                                                                                                                                        & 1ar           & ID asociado                                                            \\\hline
        <LETRA>                        & [A-Za-z]                                                                                                                                                                                                                                                                                                                                                                                                                                                                                                                                                     & a             & ID asociado                                                            \\\hline
        <DIGITO>                       & [0-9]                                                                                                                                                                                                                                                                                                                                                                                                                                                                                                                                                        & 7             & Valor numérico, lexema asociado                                        \\\hline
        <TIPO>                         & modelo                                                                                                                                                                                                                                                                                                                                                                                                                                                                                                                                                       & modelo        & ID asociado                                                            \\\hline
        <OPERACION>                    & graficar2d | graficar3d | pesomolecular                                                                                                                                                                                                                                                                                                                                                                                                                                                                                                                      & pesomolecular & ID asociado                                                            \\\hline
        <MODELO \_MOLECULAR>           & (\{ELEMENTO \_QUIMICO\} | \{ELEMENTO \_QUIMICO\} \{VALENCIA\} | \{ELEMENTO\} \{GRUPO \_FUNCIONAL\} | \{ELEMENTO\} \{GRUPO \_FUNCIONAL\} \{ENLACE\} | \{ELEMENTO\} \{ENLACE\})                                                                                                                                                                                                                                                                                                                                                                                & CH3(CH3)CHH   & ID asociado                                                            \\\hline
        <COMPUESTO>                    & COMPUESTO (\{ELEMENTO \_QUIMICO\} | \{ELEMENTO \_QUIMICO\} \{VALENCIA\} | \{ELEMENTO\} \{GRUPO\_FUNCIONAL\} | \{ELEMENTO\} \{GRUPO \_FUNCIONAL\} \{ENLACE\} | \{ELEMENTO\} \{ENLACE\})                                                                                                                                                                                                                                                                                                                                                                       & CH3::         & Modelo molecular asociado, enlaces, valencias                          \\\hline
        <COMPUESTOS>                   & \{COMPUESTO\}+                                                                                                                                                                                                                                                                                                                                                                                                                                                                                                                                               & CH3::(OH)3    & Modelo molecular asociado, enlaces, valencias                          \\\hline
        <ELEMENTO>                     & \{ELEMENTO \_QUIMICO\} \{VALENCIA\}?                                                                                                                                                                                                                                                                                                                                                                                                                                                                                                                         & Ag3           & Elemento, valencia                                                     \\\hline
        <ELEMENTO \_QUIMICO>           & ( "H" | "Li" | "Na" | "K" | "Rb" | "Cs" | "Fr" | "Be" | "Mg" | "Ca" | "Sr" | "Ba" | "Ra" | "Sc" | "Y" | "Ti" | "Zr" | "Hf" | "Db" | "V" | "Nb" | "Ta" | "Ji" | "Cr" | "Mo" | "W" | "Rf" | "Mn" | "Tc" | "Re" | "Bh" | "Fe" | "Ru" | "Os" | "Hn" | "Co" | "Rh" | "Ir" | "Mt" | "Ni" | "Pd" | "Pt" | "Cu" | "Ag" | "Au" | "Zn" | "Cd" | "Hg" | "B" | "Al" | "Ga" | "In" | "Ti" | "C" | "Si" | "Ge" | "Sn" | "Pb" | "N" | "P" | "As" | "Sb" | "Bi" | "O" | "S" | "Se" | "Te" | "Po" | "F" | "Cr" | "Br" | "I" | "At" | "He" | "Ne" | "Ar" | "Kr" | "Xe" | "Rn") & I             & Elemento                                                               \\\hline
        <VALENCIA>                     & [1-9]                                                                                                                                                                                                                                                                                                                                                                                                                                                                                                                                                        & 2             & Valor                                                                  \\\hline
        <GRUPO \_FUNCIONAL>            & ( \{GRUPO \_FUNCIONAL \_INFERIOR\} \{GRUPO \_FUNCIONAL \_SUPERIOR\} | \{GRUPO \_FUNCIONAL \_SUPERIOR\} \{GRUPO \_FUNCIONAL\_INFERIOR\} | "(" \{MODELO \_GRUPO \_FUNCIONAL\} ")" | "[" {MODELO \_GRUPO \_FUNCIONAL} "]")                                                                                                                                                                                                                                                                                                                                      & (CH3)\{Ag2\}  & Grupos funcionales, grupo funcional inferior, grupo funcional superior \\\hline
        <GRUPO \_FUNCIONAL \_INFERIOR> & "[" \{MODELO \_GRUPO \_FUNCIONAL\} "]"                                                                                                                                                                                                                                                                                                                                                                                                                                                                                                                       & [CVHe3]       & Elementos, valencias                                                   \\\hline
        <GRUPO \_FUNCIONAL \_SUPERIOR> & "(" \{MODELO \_GRUPO \_FUNCIONAL\} ")"                                                                                                                                                                                                                                                                                                                                                                                                                                                                                                                       & (CVHe3)       & Elementos, valencias                                                   \\\hline
        <MODELO \_GRUPO \_FUNCIONAL>   & (\{ELEMENTO \_QUIMICO\}+ \{VALENCIA\}?)+ | (\{ELEMENTO\}+ \{ENLACE\} \{ELEMENTO\}+)+                                                                                                                                                                                                                                                                                                                                                                                                                                                                         & FeH=C3Si4     & Elementos, enlaces, valencias                                          \\\hline
        <ENLACE>                       & ("-" | "=" | ":" | "::")                                                                                                                                                                                                                                                                                                                                                                                                                                                                                                                                     & -             & Valencia                                                               \\\hline
        \caption{Tabla de Componentes Léxicos de AVISMO}
        \label{table:lexTable}
    \end{longtable}
\end{landscape}

\section{Diseño del del Analizador Léxico}

\subsection{Autómatas Finitos Deterministas}

\begin{figure}[H]
    \footnotesize
    \begin{minipage}{.5\textwidth}
        \centering
        \begin{mdframed}
            \begin{tikzpicture}[node distance = 2.5cm, on grid, auto]
                \node [state,initial] (q0) {$0$};
                \node [state,accepting] [right=of q0] (q1) {$1$};
                \node [state,accepting] [below=of q1] (q2) {$2$};

                \path[-stealth, thick]
                (q0) edge node {;} (q1)
                (q0) edge [bend right] node {:} (q2);
            \end{tikzpicture}
            \label{fig: finDeLineaAutomata}
        \end{mdframed}
        \caption{Automata del patrón para el token <FIN\_DE\_LINEA>}
    \end{minipage}\hspace{1cm}
    \begin{minipage}{0.5\textwidth}
        \centering
        \begin{mdframed}
            \begin{tikzpicture}[node distance = 2.5cm, on grid, auto]
                \node [state, initial] (q0) {$0$};
                \node [state, accepting] [right=of q0] (q1) {$1$};
                \node [state, accepting] [below=of q1] {$1$};

                \path[-stealth, thick]
                (q0) edge node {defina} (q1)
                (q0) edge [bend right] node {como} (q2);
            \end{tikzpicture}
        \end{mdframed}
        \label{fig: palabraReservada}
        \captionof{figure}{Automata del patrón para el token <PALABRAS\_RESERVADA>}
    \end{minipage}
\end{figure}


\begin{figure}[H]
    \footnotesize
    \begin{minipage}{0.5\textwidth}
        \begin{mdframed}
            \begin{tikzpicture}[node distance = 2.5cm, auto]
                \node [state, initial] (q0) {0};
                \node [state, accepting] [right=of q0] (q1) {$1$};
                \node [state] [below=of q0] (q2) {$2$};
                \node [state, accepting] [right=of q2] (q3) {$3$};

                \path[-stealth, thick]
                (q0) edge node {[A-Za-z]} (q1)
                (q0) edge node {<LETRA>} (q2)
                (q2) edge node {<IDCONT>} (q3);
            \end{tikzpicture}
        \end{mdframed}
        \label{fig: idAutomata}
        \caption{Automata del patrón para el token <ID>}
    \end{minipage}\hspace{1cm}
    \begin{minipage}{0.5\textwidth}
        \begin{mdframed}
            \begin{tikzpicture}[node distance = 1.5cm, auto]
                \node [state, initial] (q0) {$0$};
                \node [state, accepting] [right=of q0] (q1) {$1$};
                \node [state] [below=of q1] (q2) {$2$};
                \node [state, accepting] [above left=of q0] (q4) {$4$};
                \node [state] [below left=of q0] (q5) {$5$};


                \path[-stealth, thick]
                (q0) edge node {[A-Za-z]} (q1)
                (q0) edge node [left] {<LETRA>} (q2)
                (q2) edge [bend right] node [right]{<IDCONT>} (q0)
                (q0) edge [bend right] node[right] {[0-9]} (q4)
                (q0) edge [bend right] node [left] {<DIGITO>} (q5)
                (q5.east) edge [bend right] node [below] {<IDCONT>} (q0);


            \end{tikzpicture}
        \end{mdframed}
        \label{fig: idContAutomata}
        \caption{Automata del patrón para el token <IDCONT>}
    \end{minipage}
\end{figure}

\begin{figure}[H]
    \footnotesize
    \begin{minipage}{0.5\textwidth}
        \begin{mdframed}
            \begin{tikzpicture}[node distance = 1.5cm, on grid, auto]
                \node [state,initial] (q0) {$0$};
                \node [state,accepting] [right=of q1] (q1) {$1$};

                \path[-stealth,thick]
                (q0) edge node {=} (q1);
            \end{tikzpicture}
        \end{mdframed}
        \label{fig: asigAutomata}
        \caption{Automata del patrón para el token <ASIGNACION>}
    \end{minipage}\hspace{1cm}
    \begin{minipage}{0.5\linewidth}
        \begin{mdframed}
            \begin{tikzpicture}[node distance = 0cm, on grid ,auto]
                \node [state,initial] (q0) {$0$};
                \node [state,accepting] [right=of q1] (q1) {$1$};

                \path[-stealth,thick]
                (q0) edge node {[0-9]} (q1);
            \end{tikzpicture}
        \end{mdframed}
        \label{fig: letraAutomata}
        \caption{Automata del patrón para el token <LETRA>}
    \end{minipage}
\end{figure}

\begin{center}
    \begin{figure}[H]
        \footnotesize
        \begin{mdframed}
            \begin{tikzpicture}[node distance = 0cm, on grid ,auto]
                \node [state,initial] (q0) {$0$};
                \node [state,accepting] [right=of q1] (q1) {$1$};

                \path[-stealth,thick]
                (q0) edge node {[0-9]} (q1);
            \end{tikzpicture}
        \end{mdframed}
        \label{fig: digitoAutomata}
        \caption{Automata del patrón para el token <DIGITO>}
    \end{figure}
\end{center}

\subsection{Tabla de Símbolos}

\begin{figure}[H]
    \begin{tikzpicture}[
            node distance=2cm and 3cm,
            ID/.style={rectangle, rounded corners, draw=black, thick, minimum size=10mm},
            ATT/.style={rectangle, rounded corners, draw=red!60, thick, minimum size=10mm},
        ]
        \draw node at (0,1.6)   {Identificador};
        \draw node at (7.5, 1.6)  {Atributo};

        \node[ID]   (var1)                  {var1};
        \node[ID]   (Big)   [below=of var1] {Big};

        \node[ATT]  (str1)  [right=of var1]  {str, val:"chungus", esMutable:false};
        \node[ATT]  (int1)  [right=of Big]   {int, val:"12",      esMutable:false};

        \draw[->, very thick] (var1.east) to (str1.west);
        \draw[->, very thick] (Big.east) to (int1.west);

        \node[draw=black, thick, fit={(var1) (Big)}, inner sep=10pt] (box) {};
        \node[draw=black, thick, fit={(str1) (int1)}, inner sep=10pt] (box) {};
    \end{tikzpicture}
    \label{fig: tablaDeSimbolos}
    \caption{Tabla de símbolos implementada como un diccionario}
\end{figure}

\section{Implementación del Analizador Léxico}


Como se ha mencionado anteriormente, la implementación lexica del proyecto fue inspirada por la implementación de Calc++ por bwasti y adaptada para la gramática de AVISMO \cite{wasti_bwastibison-example-calc-_2020}. 

Para compilar el programa, primero que todo se tiene que ejecutar \textit{make clean} por la línea dentro del directorio \textit{code}, Esto se hace con el propósito de evitar errores de compilación.
Entonces, se ejecuta el comando \textit{make}, esto compilará todas las dependencias necesarias, en particular, los archivos de contiene el léxico de Flex (con todos los archivos con extensión .cc, .hh y .ll).
Además, el comando anterior compila todos los programas en código objeto (.o) que se crean para el programa \textit{driver} (el cual está encargado de abrir archivos de entrada e instanciar el analizador), parser y scanner.
Posterior a esto es que se puede ejecutar \textit{./avismo fileName.txt} el cual para nuestro caso seria el siguiente: 

\begin{center}
\textbf{\textit{./avismo test\_prog.txt}}
\end{center}

Al ejecutarse el comando anterior, el programa procede a leer \textbf{cada caracter} del programa e identificar si una serie de caracteres sigue un patrón que forma parte del lenguaje AVISMO. Al encontrar un patrón reconocido, tales como un identificador o modelo molecular, lo clasifica con un \textit{token} correspondiente, lo emprime en el archivo de \href{https://github.com/aramis-matos/comp4999_compilers_project/blob/master/code/output.txt}\textit{output.txt} y lo devuelve al analizador sintáctico. Note, que el patrón de identificador reconoce palabra reservadas también. Esto crea ambiguedad semántica debido a que la gramatica no tiene un mecanismo para diferenciar entre una palabra reservada y un identificador. Por esta razón, si una serie de caracteres se identifica como un lexema de categoría identificador, se compara con los valores ya existentes del diccionario \textit{variables}, el cual es un miembro de la clase \textit{driver}. Al inicializar un objeto \textit{driver}, cuyo constructor está localizado en \href{https://github.com/aramis-matos/comp4999_compilers_project/blob/master/code/driver.cc}{\textit{driver.cc}}, este se encarga de abrir el archivo de
palabras reservadas (\href{https://github.com/aramis-matos/comp4999_compilers_project/blob/master/code/keywords.txt}{\textit{keywords.txt}}) y añadir las palabras reservadas antes que cualquier variable se pueda inicializar. Mas aún, a las palabras reservadas se les asigna el valor de la cadena vacía. Esto se hace con el propósito de poder diferenciar entre palabras reservadas e identificadores, ya que al nivel sintáctico, no es posible asignarle a una identificador una cadena vacía, como se puede apreciar a continuación:
\begin{lstlisting}
{ID} {	
  std::string text(yytext);
  if (drv.variables.find(text) != drv.variables.end() && drv.variables[text] == "") {
    if (drv.variables[text] == "") {
      format_output("PALABRA_RESERVADA",yytext,loc);
      return yy::parser::make_PALABRA_RESERVADA (yytext,loc);
    }
  }
	format_output("ID",yytext,loc);
	return yy::parser::make_ID(text,loc);
}
\end{lstlisting}
Con el propósito de visualizar los lexemas generados por el \textit{scanner}, colocado en el archivo \href{https://github.com/aramis-matos/comp4999_compilers_project/blob/master/code/scanner.ll}{\textit{scanner.ll}}, se utiliza la siguiente función:
\begin{lstlisting}
std::ofstream& file("output.txt");
void format_output (std::string token,const char* yytext, yy::location& loc) {
    file << "(" << "<" << token << ">," << std::string(yytext) << "," << loc << ")" << std::endl;
}
\end{lstlisting}
En el caso de un error léxico, se ejecuta el siguiente código:
\begin{lstlisting}
. {
  file << "CARACTER INVALIDO " << std::string(yytext) << "," << loc << std::endl;
}
\end{lstlisting}
Los patrones que se utilizan para este archivo \href{https://github.com/aramis-matos/comp4999_compilers_project/blob/master/code/scanner.ll}{\textit{scanner.ll}} (líneas 30-71) son una adaptación de la gramática en la tabla \ref{table:lexTable}. Note las variables \textit{yytext} y \textit{loc} en el código anterior. 
\textit{loc} contiene la referencia a la memoria de la variable \textit{location} de la clase \textit{driver}, cuyo propósito es retornar el valor de la linea en donde se encuentra un lexema dado.
\textit{yytext} contiene el lexema que fue aceptado por un patrón.
\textit{loc} y \textit{yytext} se utilizan para imprimir la aceptación de una cadena de caracteres como un lexema de un \textit{token} o para gestión de errores. Por ejemplo, el siguiente código genera una entrada en el archivo \href{https://github.com/aramis-matos/comp4999_compilers_project/blob/master/code/output.txt}{\textit{output.txt}} que decalara la cadena de caracteres que se aceptó como una sentencia.
\begin{lstlisting}
format_output("SENTENCIAS",yytext,loc);
\end{lstlisting}

Todas las definiciones de patrones (con la exepción del patrón que maneja la gestión de errores) contienen la función de \textit{format\_output} para imprimir la aceptación de una serie de caracteres como un lexema de un patrón en particular.
Además, como se ha mencionado anteriormente, una vez se hace la aceptación, se devuelve el \textit{token} al analizador sintáctico.

\begin{lstlisting}
{SENTENCIAS} {
    format_output("SENTENCIAS",yytext,loc);\\
    return yy::parser::make_SENTENCIAS(yytext,loc);\\
}
\end{lstlisting}

