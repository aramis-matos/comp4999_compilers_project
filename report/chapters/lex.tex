\section{Especificación y Diseño del Analizador Léxico}

\subsubsection{Gramatica a Expresiones Regulares}

\begin{enumerate}
    \item SENTENCIAS = SENTENCIA FIN\_DE\_LINEA SENTENCIAS
    \item FIN\_DE\_LINEA = (;|:)
    \item SENTENCIA = \textit{defina} ID \textit{como} TIPO | ID \textit{=} MODELO\_MOLECULAR | OPERACION \textit{(\textnormal{ID})}
    \item ID = \textit{[A-Za-z]} | LETRA IDCONT
    \item IDCONT = \textit{[A-Za-z]} | LETRA IDCONT | \textit{[0-9]} | DIGITO IDCONT
    \item LETRA = \textit{[A-Za-z]}
\end{enumerate}

\begin{table}
    \footnotesize
    \begin{tabularx}{\linewidth}{|X|X|X|X|}
        \hline
        Lexema & \textit{Token}        & Patrón                                                            & Atributo                                          \\\hline
        ;      & <FIN\_DE\_ LINEA>     & ; | :                                                             & Indica fin de Línea                               \\\hline
        defina & <PALABRA\_ RESERVADA> & defina | como                                                     & Indica declaración de una variable                \\\hline
        VaR1   & <ID>                  & [A-Za-z] | <letra> <idcont>                                       & Apuntador a la tabla de símbolos                  \\\hline
        X1     & <IDCONT>              & \textit{[A-Za-z]} | LETRA IDCONT | \textit{[0-9]} | DIGITO IDCONT & Permite que los identificadores contengan números \\\hline
        =      & <ASIGNACION>          & =                                                                 & Asigna un <MODELO\_MOLECULAR a un identificador   \\\hline
        X1y2   & <ID>                  & [A-Za-z] | <letra> <idcont>                                       & Indica declaración de una variable                \\\hline
    \end{tabularx}
    \label{table: lexTable}
    \caption{Definición de Patrones}
\end{table}



\section{Implementación del Analizador Léxico}