\section{\textit{grammar.py}}
Un código fuente pasa por al menos dos fases:
\begin{enumerate}
	\item Análisis léxico
	\item Análisis sintáctico
\end{enumerate}

Como se ha mencionado antes, en esta primera fase se evalua el código fuente caracter a caracter.
Este se tokeniza, es decir se le otorga una categoría sintáctica, y se devuelve al analizador sintáctico.
Ahora, la cuestión es, cuál es el propósito del analizador sintáctico?
El analizador sintáctico recibe una lista de \textit{tokens} del analizador léxico y las convierte, atravez de reglas de producción, en sentencias gramaticales del lenguaje en cuestión.
Relas de producción tienen la siguiente forma: (\textbf{Regla} : \textit{Definición}) donde \textbf{Regla} es un no terminal y \textit{Definición} es una serie de 0 o mas terminales o no terminales.
Un terminal se define como un \textit{token} y un no terminal es un una regla gramatical en si.

Se utiliza PLY \cite{noauthor_ply_nodate} para el análisis sintáctico. En particular, su implementación de \textit{yacc} \cite{noauthor_man_nodate}.
En el archivo \href{https://github.com/aramis-matos/comp4999_compilers_project/blob/master/code/python_remake/grammar.py}{\textit{grammar.py}} se puede apreciar que la todas de las reglas del lenguaje AVISMO, con la exepción de relas que fueron utilizadas en analizador léxico, fueron adaptadas.
En PLY, una regla gramatical se define como una función en Python cuyo nombre es \textbf{p\_} seguido del nombre de la regla de producción, por ejemplo:
\begin{figure}[H]
	\begin{minted}[breaklines,breakanywhere,linenos,escapeinside=!!]{python}
def p_s(p):
'''s : INICIO sentencias FIN''' !\label{rule}!
	\end{minted}
	\caption{Ejemplo de una regla de producción en PLY}
	\label{fig: grammarRuleExample}
\end{figure}
Note que el argumento \textit{p} es una lista que contiene objetos \textit{LexToken}.
Los terminales tienen una variable de valor asignada mientras que los no terminales no.
Cada objeto \textit{LexToken} tiene una posición léxica (como una variable miembro llamada \textit{lexpos}) y la línea dentro del código fuente  (como una variable miembro llamada \textit{lineno}).
La manera de definir la regla de producción se puede ver en la figura \ref{fig: grammarRuleExample}, linea \ref{rule}.
Esta sigue el formato previamente establecido pero con un detalle importante.
Los no terminales están escritos en letras minúsculas y los terminales en mayúsculas.
Esto se hizo con el motivo de clarificar en que categoría, si terminal o no terminal, es clasificada cada item en la regla de producción.
Más aún, la documentación de \textit{PLY} sugiere esta convención.

Toda gramática parte desde un axioma y \textit{PLY} sigue este principio.
Por defacto, \textit{PLY} asume que la primera regla que se define en el archivo de \textit{grammar.py} es el axioma del la gramatica.
Sin embargo, es preferible que se defina un axioma explícito.
En \textit{PLY}, si se le asigna a la variable \textit{start} el nombre de la regla de producción como una cadena de caracteres, como se hace a continuación, \mintinline{python}{start = "s"}, \textit{PLY} explícitamente comienza la derivación desde esa regla. Declarar el axioma explícitamente tiene dos ventajas:
\begin{itemize}
	\item Claridad en el código
	\item Eliminación de errores por tokens no utilizados
\end{itemize}

Debido a que no se está implementando la funcionalidad del lenguaje AVISMO, las reglas de producción no tienen código relevante. Sin embargo, todas las reglas de producción en \textit{grammar.py} ejecutan una función llamda \textit{format\_expr} que guarda información acerca de la regla gramatical que se utilizó en la derivación del código fuente.
Esto se hace con intenciones pedagógicas.
El código de \textit{format\_expr} se presenta a continuación:
\begin{figure}[H]
	\begin{minted}[breaklines,breakanywhere,linenos]{python}
def format_expr(p):
    types = [x.type for x in p.slice]
    rule = f"{types[0]} --> "
    for val in types[1:]:
        rule += f"{val} "
    rules.append([rule])
\end{minted}
	\caption{Código para guardar información acerca de las derivaciones}
	\label{fig: formatExpr}
\end{figure}

\section{\textit{tester.py}}

Para poder correr \textit(grammar.py) en un archivo escrito en AVISMO, es necesario invocar de correr el programa \textit{tester.py} de la siguiente manera:

\begin{minted}{Bash}[breaklines,Breakanywhere]
	python3 tester.py archivo 
\end{minted}




