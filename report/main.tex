\documentclass[12pt]{report}
\usepackage[acronym,automake]{glossaries-extra}
\usepackage{hyperref}
\usepackage{unicode}
\usepackage{times}
\usepackage{apacite}
\usepackage{tabularx}
\usepackage[doublespacing]{setspace}
\usepackage{tikz,pgfplots}
\usepackage{mdframed}
\usepackage{caption}
\usepackage{float}
\usepackage{longtable}
\usepackage{pdflscape}
\usepackage{listings}
\usepackage{minted}
\lstset{
basicstyle=\small\ttfamily,
columns=flexible,
breaklines=true
}



\usetikzlibrary{positioning,arrows,automata}
\usetikzlibrary{arrows.meta}
\usetikzlibrary{fit}
\pgfplotsset{compat=1.16}

\setabbreviationstyle[acronym]{long-short}
\makeglossaries
\title{Diseño e Implementación de un Analizador Léxico y Analizador Semántico para el Lenguaje AVISMO}
\author{Aramis E. Matos \\ Lenier Gerena \\ Angel Berrios Pellot}
\date{Segundo Semestre, 2022-2023}
\begin{document}
\renewcommand{\acronymname}{Acrónimos}
\renewcommand{\bibname}{Referencias Bibliográficas}
\renewcommand{\contentsname}{Tabla de Contenido}
\renewcommand{\chaptername}{Capítulo}
\renewcommand{\figurename}{Figura}
\renewcommand{\tablename}{Tabla}
\renewcommand{\listingscaption}{Listado}

\maketitle
\tableofcontents

\chapter{Introducción}

\textbf{TEMP}: Este reporte está diseñando un analizador léxico y semántico para el lenguaje AVISMO \cite{narciso_farias_gramatica_2012}

\chapter{Analizador Léxico}

\section{Especificación y Diseño del Analizador Léxico}

\subsubsection{Gramatica a Expresiones Regulares}

\begin{enumerate}
    \item SENTENCIAS = SENTENCIA FIN\_DE\_LINEA SENTENCIAS
    \item FIN\_DE\_LINEA = (;|:)
    \item SENTENCIA = \textit{defina} ID \textit{como} TIPO | ID \textit{=} MODELO\_MOLECULAR | OPERACION \textit{(\textnormal{ID})}
    \item ID = \textit{[A-Za-z]} | LETRA IDCONT
    \item IDCONT = \textit{[A-Za-z]} | LETRA IDCONT | \textit{[0-9]} | DIGITO IDCONT
    \item LETRA = \textit{[A-Za-z]}
\end{enumerate}

\begin{table}
    \footnotesize
    \begin{tabularx}{\linewidth}{|X|X|X|X|}
        \hline
        Lexema & \textit{Token}        & Patrón                                                            & Atributo                                          \\\hline
        ;      & <FIN\_DE\_ LINEA>     & ; | :                                                             & Indica fin de Línea                               \\\hline
        defina & <PALABRA\_ RESERVADA> & defina | como                                                     & Indica declaración de una variable                \\\hline
        VaR1   & <ID>                  & [A-Za-z] | <letra> <idcont>                                       & Apuntador a la tabla de símbolos                  \\\hline
        X1     & <IDCONT>              & \textit{[A-Za-z]} | LETRA IDCONT | \textit{[0-9]} | DIGITO IDCONT & Permite que los identificadores contengan números \\\hline
        =      & <ASIGNACION>          & =                                                                 & Asigna un <MODELO\_MOLECULAR a un identificador   \\\hline
        X1y2   & <ID>                  & [A-Za-z] | <letra> <idcont>                                       & Indica declaración de una variable                \\\hline
    \end{tabularx}
    \label{table: lexTable}
    \caption{Definición de Patrones}
\end{table}



\section{Implementación del Analizador Léxico}

\chapter{Implementación del Analizador Sintáctico}

\section{\textit{grammar.py}}
Un código fuente pasa por al menos dos fases:
\begin{enumerate}
	\item Análisis léxico
	\item Análisis sintáctico
\end{enumerate}

Como se ha mencionado antes, en esta primera fase se evalua el código fuente caracter a caracter.
Este se tokeniza, es decir se le otorga una categoría sintáctica, y se devuelve al analizador sintáctico.
Ahora, la cuestión es, cuál es el propósito del analizador sintáctico?
El analizador sintáctico recibe una lista de \textit{tokens} del analizador léxico y las convierte, atravez de reglas de producción, en sentencias gramaticales del lenguaje en cuestión.
Relas de producción tienen la siguiente forma: (\textbf{Regla} : \textit{Definición}) donde \textbf{Regla} es un no terminal y \textit{Definición} es una serie de 0 o mas terminales o no terminales.
Un terminal se define como un \textit{token} y un no terminal es un una regla gramatical en si.

Se utiliza PLY \cite{noauthor_ply_nodate} para el análisis sintáctico. En particular, su implementación de \textit{yacc} \cite{noauthor_man_nodate}.
En el archivo \href{https://github.com/aramis-matos/comp4999_compilers_project/blob/master/code/python_remake/grammar.py}{\textit{grammar.py}} se puede apreciar que la todas de las reglas del lenguaje AVISMO, con la exepción de relas que fueron utilizadas en analizador léxico, fueron adaptadas.
En PLY, una regla gramatical se define como una función en Python cuyo nombre es \textbf{p\_} seguido del nombre de la regla de producción, por ejemplo:
\begin{figure}[H]
	\begin{minted}[breaklines,breakanywhere,linenos,escapeinside=!!]{python}
def p_s(p):
'''s : INICIO sentencias FIN''' !\label{rule}!
	\end{minted}
	\caption{Ejemplo de una regla de producción en PLY}
	\label{fig: grammarRuleExample}
\end{figure}
Note que el argumento \textit{p} es una lista que contiene objetos \textit{LexToken}.
Los terminales tienen una variable de valor asignada mientras que los no terminales no.
Cada objeto \textit{LexToken} tiene una posición léxica (como una variable miembro llamada \textit{lexpos}) y la línea dentro del código fuente  (como una variable miembro llamada \textit{lineno}).
La manera de definir la regla de producción se puede ver en la figura \ref{fig: grammarRuleExample}, linea \ref{rule}.
Esta sigue el formato previamente establecido pero con un detalle importante.
Los no terminales están escritos en letras minúsculas y los terminales en mayúsculas.
Esto se hizo con el motivo de clarificar en que categoría, si terminal o no terminal, es clasificada cada item en la regla de producción.
Más aún, la documentación de \textit{PLY} sugiere esta convención.

Toda gramática parte desde un axioma y \textit{PLY} sigue este principio.
Por defacto, \textit{PLY} asume que la primera regla que se define en el archivo de \textit{grammar.py} es el axioma del la gramatica.
Sin embargo, es preferible que se defina un axioma explícito.
En \textit{PLY}, si se le asigna a la variable \textit{start} el nombre de la regla de producción como una cadena de caracteres, como se hace a continuación, \mintinline{python}{start = "s"}, \textit{PLY} explícitamente comienza la derivación desde esa regla. Declarar el axioma explícitamente tiene dos ventajas:
\begin{itemize}
	\item Claridad en el código
	\item Eliminación de errores por tokens no utilizados
\end{itemize}

Debido a que no se está implementando la funcionalidad del lenguaje AVISMO, las reglas de producción no tienen código relevante. Sin embargo, todas las reglas de producción en \textit{grammar.py} ejecutan una función llamda \textit{format\_expr} que guarda información acerca de la regla gramatical que se utilizó en la derivación del código fuente.
Esto se hace con intenciones pedagógicas.
El código de \textit{format\_expr} se presenta a continuación:
\begin{figure}[H]
	\begin{minted}[breaklines,breakanywhere,linenos]{python}
def format_expr(p):
    types = [x.type for x in p.slice]
    rule = f"{types[0]} --> "
    for val in types[1:]:
        rule += f"{val} "
    rules.append([rule])
\end{minted}
	\caption{Código para guardar información acerca de las derivaciones}
	\label{fig: formatExpr}
\end{figure}

\section{\textit{tester.py}}

Para poder correr \textit(grammar.py) en un archivo escrito en AVISMO, es necesario invocar de correr el programa \textit{tester.py} de la siguiente manera:

\begin{minted}{Bash}[breaklines,Breakanywhere]
	python3 tester.py archivo 
\end{minted}






\chapter{Conclusiones y Recomendaciones}

Inicialmente, el analizador léxico de este proyecto fue escrito con \textit{Flex} \cite{noauthor_flex_nodate} en \textit{C++}. 
Esto se hizo porque nosotros no habíamos escrito en \textit{C++} en mucho tiempo y deseábamos elaborar un proyecto extenso en él para mejorar nuestro entendimiento del lenguaje. 
Sin embargo, se tuvo que abandonar este camino debido a una combinación de limitaciones de tiempo, documentación pobre, pocos recursos de donde tomar inspiración, etc. 
Debido a esta situación, se tuvo que re-escribir el analizador léxico en PLY, la cual tiene documentación mejor, recursos extensos, buenos ejemplos, entre otros beneficios. 
De este cambio se aprendieron varias lecciones. 
Entre ellas, la importancia de utilizar la herramienta más apropiada para el trabajo. 
En el desarrollo de \textit{software}, es importante utilizar herramientas que tengan una base amplia de soporte, independientemente de las metas personales de los diseñadores. 

En conclusión, este proyecto fue una experiencia fascinante y despertadora.
Como programadores, los compiladores y interpretadores son nuestras herramientas de uso diario, como es el martillo para un carpintero.
Aveces se nos olvida que los lenguaje de programación están diseñados para ser escritos y entendidos por humanos porque su estructura parece tan disimilar a las lenguas naturales. 
Gracias a esta experiencia, dimos un paso hacia atrás y pudimos apreciar lo complejo que es diseñar un analizador léxico y sintáctico.
Creemos que jamás perderemos la paciencia con un error de compilación.
Sinó nos sentiremos agradecidos que algún programador tomo el tiempo de crear la herramientas que nos provee un sueldo. 
No creemos que sea una exageración decir que gracias a la labor colaborativa de muchos académicos a traves del tiempo, han cambiado el mundo, una línea de código a la vez.

\listoffigures

\listoftables

\bibliographystyle{apacite}
\bibliography{References}
\printglossary[type=\acronymtype]
\end{document}