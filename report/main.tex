\documentclass[12pt]{report}
\usepackage[acronym,automake]{glossaries-extra}
\usepackage{hyperref}
\usepackage{unicode}
\usepackage{times}
\usepackage{apacite}
\usepackage{tabularx}
\usepackage[doublespacing]{setspace}
\usepackage{tikz,pgfplots}
\usepackage{mdframed}
\usepackage{caption}
\usepackage{float}
\usepackage{longtable}
\usepackage{pdflscape}
\usepackage[chapter]{minted}
\usepackage{listings}
\lstset{
basicstyle=\small\ttfamily,
columns=flexible,
breaklines=true
}
\newcommand {\scaledimage}[1] {

    \includegraphics[width=\textwidth,height=0.8\textheight,keepaspectratio]{#1}
}

\usetikzlibrary{positioning,arrows,automata}
\usetikzlibrary{arrows.meta}
\usetikzlibrary{fit}
\pgfplotsset{compat=1.16}
\usemintedstyle{perldoc}

\setabbreviationstyle[acronym]{long-short}
\makeglossaries
\title{Diseño e Implementación de un Analizador Léxico y Analizador Sintáctico para el Lenguaje AVISMO}
\author{Aramis E. Matos \\ Lenier Gerena \\ Angel Berrios Pellot}
\date{Segundo Semestre, 2022-2023}
\begin{document}
\renewcommand{\acronymname}{Acrónimos}
\renewcommand{\bibname}{Referencias Bibliográficas}
\renewcommand{\contentsname}{Tabla de Contenido}
\renewcommand{\chaptername}{Capítulo}
\renewcommand{\figurename}{Figura}
\renewcommand{\tablename}{Tabla}
\renewcommand{\listingscaption}{Listado}
\renewcommand{\listtablename}{Lista de Tablas}
\renewcommand{\listfigurename}{Lista de Figuras}
% \renewcommand{\lstlistlistingname}{def}

\maketitle
\tableofcontents

\chapter{Introducción}

\textit{La visualización molecular puede ser considerada como una de las áreas mas importante dentro de la bioinformática. Entre sus aplicaciones mas relevantes se destacan el diseño de nuevo fármacos...} \cite{narciso_farias_gramatica_2012}.
Este enunciado fue escrito hace mas de una década. Sin embargo, hoy día en un mundo pospandemia, reconocemos que tan sabio fue.
El desarollo de la vacuna contra el COVID-19 tan rápido fue gracias a herramientas de visualización como el Ambiente de visualización Molecular (AVISMO) \cite{narciso_farias_gramatica_2012}
El propósito de este proyecto es definir el automata de estado finito del lenguaje AVISMO, los patrones el cual caracterizan los lexemas del lenguaje, los atributos de los lexemas y la implementación del analizador léxico y sintáctico del lenguaje AVISO en C++.
La implementación léxica se desarolló utilizando GNU Flex \cite{noauthor_flex_nodate} con la implementación de Calc++ por bwasti \cite{wasti_bwastibison-example-calc-_2020} como base.
La implementación sintáctica se desarolló en GNU Bison \cite{noauthor_bison_nodate} con la implementación de Calc++ por bwasti \cite{wasti_bwastibison-example-calc-_2020} como base.


\chapter{Reglas de Producción}

\section{Gramática del Lenguaje AVISMO}

\begin{itemize}
    \item <SENTENCIAS> ::= <FIN\_DE\_LINEA> <SENTENCIAS> | <SENTENCIA> <FIN\_DE\_LINEA>
    \item <FIN\_DE\_LINEA> ::= ":" | ";"
    \item <SENTENCIA> ::= "defina" <ID> "como" <TIPO> | <ID> "="  <MODELO\_MOLECULAR> | <OPERACION> "(" <ID> ")"
    \item <ID> ::= "A" | "B" | "C" | "D" | "E" | "F" | "G" | "H" | "I" | "J" | "K" | "L" | "M" | "N" | "O" | "P" | "Q" | "R" | "S" | "T" | "U" | "V" | "W" | "X" | "Y" | "Z" | "a" | "b" | "c" | "d" | "e" | "f" | "g" | "h" | "i" | "j" | "k" | "l" | "m" | "n" | "o" | "p" | "q" | "r" | "s" | "t" | "u" | "v" | "w" | "x" | "y" | "z" | <LETRA> <IDCONT>
    \item <IDCONT> ::= "A" | "B" | "C" | "D" | "E" | "F" | "G" | "H" | "I" | "J" | "K" | "L" | "M" | "N" | "O" | "P" | "Q" | "R" | "S" | "T" | "U" | "V" | "W" | "X" | "Y" | "Z" | "a" | "b" | "c" | "d" | "e" | "f" | "g" | "h" | "i" | "j" | "k" | "l" | "m" | "n" | "o" | "p" | "q" | "r" | "s" | "t" | "u" | "v" | "w" | "x" | "y" | "z" | <LETRA> <IDCONT> | "0" | "1" | "2" | "3" | "4" | "5" | "6" | "7" | "8" | "9" | <DIGITO> <IDCONT>
    \item <LETRA> ::= "A" | "B" | "C" | "D" | "E" | "F" | "G" | "H" | "I" | "J" | "K" | "L" | "M" | "N" | "O" | "P" | "Q" | "R" | "S" | "T" | "U" | "V" | "W" | "X" | "Y" | "Z" | "a" | "b" | "c" | "d" | "e" | "f" | "g" | "h" | "i" | "j" | "k" | "l" | "m" | "n" | "o" | "p" | "q" | "r" | "s" | "t" | "u" | "v" | "w" | "x" | "y" | "z"
    \item <DIGITO> ::= "0" | "1" | "2" | "3" | "4" | "5" | "6" | "7" | "8" | "9"
    \item <TIPO> ::= "modelo"
    \item <OPERACION> ::= "graficar2d" |"graficar3d" | "pesomolecular"
    \item <MODELO\_MOLECULAR> ::= "H" | "Li" | "Na" | "K" | "Rb" | "Cs" | "Fr" | "Be" | "Mg" | "Ca" | "Sr" | "Ba" | "Ra" | "Sc" | "Y" | "Ti" | "Zr" | "Hf" | "Db" | "V" | "Nb" | "Ta" | "Ji" | "Cr" | "Mo" | "W" | "Rf" | "Mn" | "Tc" | "Re" | "Bh" | "Fe" | "Ru" | "Os" | "Hn" | "Co" | "Rh" | "Ir" | "Mt" | "Ni" | "Pd" | "Pt" | "Cu" | "Ag" | "Au" | "Zn" | "Cd" | "Hg" | "B" | "Al" | "Ga" | "In" | "Ti" | "C" | "Si" | "Ge" | "Sn" | "Pb" | "N" | "P" | "As" | "Sb" | "Bi" | "O" | "S" | "Se" | "Te" | "Po" | "F" | "Cr" | "Br" | "I" | "At" | "He" | "Ne" | "Ar" | "Kr" | "Xe" | "Rn" | <ELEMENTO\_QUIMICO> <VALENCIA> | <ELEMENTO> <GRUPO\_FUNCIONAL> | <COMPUESTO> <ELEMENTO> <GRUPO\_FUNCIONAL> | <COMPUESTO> <COMPUESTO>
    \item <COMPUESTO> ::= "H" | "Li" | "Na" | "K" | "Rb" | "Cs" | "Fr" | "Be" | "Mg" | "Ca" | "Sr" | "Ba" | "Ra" | "Sc" | "Y" | "Ti" | "Zr" | "Hf" | "Db" | "V" | "Nb" | "Ta" | "Ji" | "Cr" | "Mo" | "W" | "Rf" | "Mn" | "Tc" | "Re" | "Bh" | "Fe" | "Ru" | "Os" | "Hn" | "Co" | "Rh" | "Ir" | "Mt" | "Ni" | "Pd" | "Pt" | "Cu" | "Ag" | "Au" | "Zn" | "Cd" | "Hg" | "B" | "Al" | "Ga" | "In" | "Ti" | "C" | "Si" | "Ge" | "Sn" | "Pb" | "N" | "P" | "As" | "Sb" | "Bi" | "O" | "S" | "Se" | "Te" | "Po" | "F" | "Cr" | "Br" | "I" | "At" | "He" | "Ne" | "Ar" | "Kr" | "Xe" | "Rn" | <ELEMENTO\_QUIMICO> <VALENCIA> | <ELEMENTO> <GRUPO\_FUNCIONAL> | <ELEMENTO> <GRUPO\_FUNCIONAL> <ENLACE> | <ELEMENTO> <ENLACE>
    \item <COMPUESTOS> ::= <COMPUESTO> <COMPUESTO> | <COMPUESTOS>
    \item <ELEMENTO> ::= "H" | "Li" | "Na" | "K" | "Rb" | "Cs" | "Fr" | "Be" | "Mg" | "Ca" | "Sr" | "Ba" | "Ra" | "Sc" | "Y" | "Ti" | "Zr" | "Hf" | "Db" | "V" | "Nb" | "Ta" | "Ji" | "Cr" | "Mo" | "W" | "Rf" | "Mn" | "Tc" | "Re" | "Bh" | "Fe" | "Ru" | "Os" | "Hn" | "Co" | "Rh" | "Ir" | "Mt" | "Ni" | "Pd" | "Pt" | "Cu" | "Ag" | "Au" | "Zn" | "Cd" | "Hg" | "B" | "Al" | "Ga" | "In" | "Ti" | "C" | "Si" | "Ge" | "Sn" | "Pb" | "N" | "P" | "As" | "Sb" | "Bi" | "O" | "S" | "Se" | "Te" | "Po" | "F" | "Cr" | "Br" | "I" | "At" | "He" | "Ne" | "Ar" | "Kr" | "Xe" | "Rn" | <ELEMENTO\_QUIMICO> <VALENCIA>
    \item <ELEMENTO\_QUIMICO> ::= "H" | "Li" | "Na" | "K" | "Rb" | "Cs" | "Fr" | "Be" | "Mg" | "Ca" | "Sr" | "Ba" | "Ra" | "Sc" | "Y" | "Ti" | "Zr" | "Hf" | "Db" | "V" | "Nb" | "Ta" | "Ji" | "Cr" | "Mo" | "W" | "Rf" | "Mn" | "Tc" | "Re" | "Bh" | "Fe" | "Ru" | "Os" | "Hn" | "Co" | "Rh" | "Ir" | "Mt" | "Ni" | "Pd" | "Pt" | "Cu" | "Ag" | "Au" | "Zn" | "Cd" | "Hg" | "B" | "Al" | "Ga" | "In" | "Ti" | "C" | "Si" | "Ge" | "Sn" | "Pb" | "N" | "P" | "As" | "Sb" | "Bi" | "O" | "S" | "Se" | "Te" | "Po" | "F" | "Cr" | "Br" | "I" | "At" | "He" | "Ne" | "Ar" | "Kr" | "Xe" | "Rn"
    \item <VALENCIA> ::= "1" | "2" | "3" | "4" | "5" | "6" | "7" | "8" | "9"
    \item <GRUPO\_FUNCIONAL> ::=
          <GRUPO\_FUNCIONAL\_INFERIOR>

          <GRUPO\_FUNCIONAL\_SUPERIOR> | <GRUPO\_FUNCIONAL\_SUPERIOR> <GRUPO\_FUNCIONAL\_INFERIOR> | "(" <MODELO\_GRUPO\_FUNCIONAL> ")" | "["  <MODELO\_GRUPO\_FUNCIONAL> "]"
    \item <GRUPO\_FUNCIONAL\_SUPERIOR> ::= "[" <MODELO\_GRUPO\_FUNCIONAL> "]"
    \item <GRUPO\_FUNCIONAL\_INFERIOR> ::= "(" <MODELO\_GRUYPO\_FUNCIONAL> ")
    \item  <MODELO\_GRUYPO\_FUNCIONAL> ::= <ENLACE> <MODELO\_MOLECULAR> | "H" | "Li" | "Na" | "K" | "Rb" | "Cs" | "Fr" | "Be" | "Mg" | "Ca" | "Sr" | "Ba" | "Ra" | "Sc" | "Y" | "Ti" | "Zr" | "Hf" | "Db" | "V" | "Nb" | "Ta" | "Ji" | "Cr" | "Mo" | "W" | "Rf" | "Mn" | "Tc" | "Re" | "Bh" | "Fe" | "Ru" | "Os" | "Hn" | "Co" | "Rh" | "Ir" | "Mt" | "Ni" | "Pd" | "Pt" | "Cu" | "Ag" | "Au" | "Zn" | "Cd" | "Hg" | "B" | "Al" | "Ga" | "In" | "Ti" | "C" | "Si" | "Ge" | "Sn" | "Pb" | "N" | "P" | "As" | "Sb" | "Bi" | "O" | "S" | "Se" | "Te" | "Po" | "F" | "Cr" | "Br" | "I" | "At" | "He" | "Ne" | "Ar" | "Kr" | "Xe" | "Rn" | <ELEMENTO\_QUIMICO> <VALENCIA> | <ELEMENTO> <GRUPO\_FUNCIONAL> | <COMPUESTO> <ELEMENTO> | <COMPUESTO> <COMPUESTO> <COMPUESTOS>
\end{itemize}


\chapter{Analizador Léxico}

\section{Especificación y Diseño del Analizador Léxico}

\subsubsection{Gramatica a Expresiones Regulares}

\begin{enumerate}
    \item SENTENCIAS = SENTENCIA FIN\_DE\_LINEA SENTENCIAS
    \item FIN\_DE\_LINEA = (;|:)
    \item SENTENCIA = \textit{defina} ID \textit{como} TIPO | ID \textit{=} MODELO\_MOLECULAR | OPERACION \textit{(\textnormal{ID})}
    \item ID = \textit{[A-Za-z]} | LETRA IDCONT
    \item IDCONT = \textit{[A-Za-z]} | LETRA IDCONT | \textit{[0-9]} | DIGITO IDCONT
    \item LETRA = \textit{[A-Za-z]}
\end{enumerate}

\begin{table}
    \footnotesize
    \begin{tabularx}{\linewidth}{|X|X|X|X|}
        \hline
        Lexema & \textit{Token}        & Patrón                                                            & Atributo                                          \\\hline
        ;      & <FIN\_DE\_ LINEA>     & ; | :                                                             & Indica fin de Línea                               \\\hline
        defina & <PALABRA\_ RESERVADA> & defina | como                                                     & Indica declaración de una variable                \\\hline
        VaR1   & <ID>                  & [A-Za-z] | <letra> <idcont>                                       & Apuntador a la tabla de símbolos                  \\\hline
        X1     & <IDCONT>              & \textit{[A-Za-z]} | LETRA IDCONT | \textit{[0-9]} | DIGITO IDCONT & Permite que los identificadores contengan números \\\hline
        =      & <ASIGNACION>          & =                                                                 & Asigna un <MODELO\_MOLECULAR a un identificador   \\\hline
        X1y2   & <ID>                  & [A-Za-z] | <letra> <idcont>                                       & Indica declaración de una variable                \\\hline
    \end{tabularx}
    \label{table: lexTable}
    \caption{Definición de Patrones}
\end{table}



\section{Implementación del Analizador Léxico}

\chapter{Analizador Sintáctico}

\section{\textit{grammar.py}}
Un código fuente pasa por al menos dos fases:
\begin{enumerate}
	\item Análisis léxico
	\item Análisis sintáctico
\end{enumerate}

Como se ha mencionado antes, en esta primera fase se evalua el código fuente caracter a caracter.
Este se tokeniza, es decir se le otorga una categoría sintáctica, y se devuelve al analizador sintáctico.
Ahora, la cuestión es, cuál es el propósito del analizador sintáctico?
El analizador sintáctico recibe una lista de \textit{tokens} del analizador léxico y las convierte, atravez de reglas de producción, en sentencias gramaticales del lenguaje en cuestión.
Relas de producción tienen la siguiente forma: (\textbf{Regla} : \textit{Definición}) donde \textbf{Regla} es un no terminal y \textit{Definición} es una serie de 0 o mas terminales o no terminales.
Un terminal se define como un \textit{token} y un no terminal es un una regla gramatical en si.

Se utiliza PLY \cite{noauthor_ply_nodate} para el análisis sintáctico. En particular, su implementación de \textit{yacc} \cite{noauthor_man_nodate}.
En el archivo \href{https://github.com/aramis-matos/comp4999_compilers_project/blob/master/code/python_remake/grammar.py}{\textit{grammar.py}} se puede apreciar que la todas de las reglas del lenguaje AVISMO, con la exepción de relas que fueron utilizadas en analizador léxico, fueron adaptadas.
En PLY, una regla gramatical se define como una función en Python cuyo nombre es \textbf{p\_} seguido del nombre de la regla de producción, por ejemplo:
\begin{figure}[H]
	\begin{minted}[breaklines,breakanywhere,linenos,escapeinside=!!]{python}
def p_s(p):
'''s : INICIO sentencias FIN''' !\label{rule}!
	\end{minted}
	\caption{Ejemplo de una regla de producción en PLY}
	\label{fig: grammarRuleExample}
\end{figure}
Note que el argumento \textit{p} es una lista que contiene objetos \textit{LexToken}.
Los terminales tienen una variable de valor asignada mientras que los no terminales no.
Cada objeto \textit{LexToken} tiene una posición léxica (como una variable miembro llamada \textit{lexpos}) y la línea dentro del código fuente  (como una variable miembro llamada \textit{lineno}).
La manera de definir la regla de producción se puede ver en la figura \ref{fig: grammarRuleExample}, linea \ref{rule}.
Esta sigue el formato previamente establecido pero con un detalle importante.
Los no terminales están escritos en letras minúsculas y los terminales en mayúsculas.
Esto se hizo con el motivo de clarificar en que categoría, si terminal o no terminal, es clasificada cada item en la regla de producción.
Más aún, la documentación de \textit{PLY} sugiere esta convención.

Toda gramática parte desde un axioma y \textit{PLY} sigue este principio.
Por defacto, \textit{PLY} asume que la primera regla que se define en el archivo de \textit{grammar.py} es el axioma del la gramatica.
Sin embargo, es preferible que se defina un axioma explícito.
En \textit{PLY}, si se le asigna a la variable \textit{start} el nombre de la regla de producción como una cadena de caracteres, como se hace a continuación, \mintinline{python}{start = "s"}, \textit{PLY} explícitamente comienza la derivación desde esa regla. Declarar el axioma explícitamente tiene dos ventajas:
\begin{itemize}
	\item Claridad en el código
	\item Eliminación de errores por tokens no utilizados
\end{itemize}

Debido a que no se está implementando la funcionalidad del lenguaje AVISMO, las reglas de producción no tienen código relevante. Sin embargo, todas las reglas de producción en \textit{grammar.py} ejecutan una función llamda \textit{format\_expr} que guarda información acerca de la regla gramatical que se utilizó en la derivación del código fuente.
Esto se hace con intenciones pedagógicas.
El código de \textit{format\_expr} se presenta a continuación:
\begin{figure}[H]
	\begin{minted}[breaklines,breakanywhere,linenos]{python}
def format_expr(p):
    types = [x.type for x in p.slice]
    rule = f"{types[0]} --> "
    for val in types[1:]:
        rule += f"{val} "
    rules.append([rule])
\end{minted}
	\caption{Código para guardar información acerca de las derivaciones}
	\label{fig: formatExpr}
\end{figure}

\section{\textit{tester.py}}



\chapter{Conclusiones y Recomendaciones}

Inicialmente, el analizador léxico de este proyecto fue escrito con \textit{Flex} \cite{noauthor_flex_nodate} en \textit{C++}. 
Esto se hizo porque nosotros no habíamos escrito en \textit{C++} en mucho tiempo y deseábamos elaborar un proyecto extenso en el para mejorar nuestro entendimiento del lenguaje. 
Sin embargo, se tuvo que abandonar este camino debido a una combinación de limitaciones de tiempo, documentación pobre, pocos recursos de donde tomar inspiración, etc. 
Debido a esta situación se tuvo que re-escribir el analizador léxico en \textit{PLY}, la cual tiene documentación mejor, recursos extensos, buen ejemplos, entre otros beneficios. 
De este cambio se aprendieron varias lecciones. 
Entre ellas la importancia de utilizar la herramienta mas apropiada para el trabajo. 
En el desarrollo de \textit{software}, es importante utilizar herramientas que tengan una base amplia de soporte, independientemente de las metas personales de los diseñadores. 

En conclusión, este proyecto fue una experiencia fascinante y despertadora.
Como programadores, los compiladores y interpretadores son nuestras herramientas de uso diario, como es el martillo para un carpintero.
Aveces se nos olvida que los lenguaje de programación están diseñadas para ser escritos y entendidos por humanos porque su estructura parece tan disimilar a las lenguas naturales. 
Gracias a esta experiencia, dimos un paso hacia atrás y pudimos apreciar lo complejo que es diseñar un analizador léxico y sintáctico.
Creemos que jamas perderemos la paciencia con un error de compilación.
Sino nos sentiremos agradecidos algún programador tomo el tiempo de crear la herramientas que no tan solo nos provee un sueldo. 
No creemos que sea una exageración decir que gracias a la labor colaborativa de muchos académicos a traves del tiempo, han cambiado el mundo, una línea de código a la vez.

\appendix

\chapter{Código de Analizador Léxico}

\begin{minted}[breaklines,breakanywhere,linenos,escapeinside=!!]{python}
import os
import sys

import ply.lex as lex
from prettytable import PrettyTable


reserved = {}

variables = {}

with open("keywords.txt", "r") as f:
    for line in f:
        val = line.strip()
        reserved[val] = val.upper()

try:
    test_file = sys.argv[1]
except IndexError:
    test_file = "test_prog.txt"

if not (os.path.exists(test_file)):
    print(
        f"The file {test_file} not found, proceeding with default\
    test_prog.txt file")
    test_file = "test_prog.txt"

tokenTable = PrettyTable()

# enumera los nombre de todos tokens que puede reconocer
tokens = [
    "FIN_DE_LINEA",
    # "LETRA",
    # "DIGITO",
    "OPERACION",
    "VALENCIA",
    "ENLACE",
    # "IDCONT",
    "ID",
    "ELEMENTO_QUIMICO",
    # "MODELO_MOLECULAR",
    # "COMPUESTO",
    # "COMPUESTOS",
    # "ELEMENTO",
    # "GRUPO_FUNCIONAL",
    # "GRUPO_FUNCIONAL_INFERIOR",
    # "GRUPO_FUNCIONAL_SUPERIOR",
    # "MODELO_GRUPO_FUNCIONAL",
    # "SENTENCIA",
    # "SENTENCIAS",
    "PARENTESIS_IZQ",
    "PARENTESIS_DER",
    "TIPO",
    # "PALABRA_RESERVADA",
    "COR_IZQ",
    "COR_DER",
    "ASIGNACION",
]

tokens = tokens + list(reserved.values())


# definiciones de los tokens y reglas de expresiones regulares
# t_COR_IZQ y t_COR_DER definen los tokens para corchetes izquierdos y
# derechos [ y ]
t_COR_IZQ = r"\["
t_COR_DER = r'\]'
# t_PARENTESIS_IZQ y t_PARENTESIS_DER definen los tokens para parentesis

# izquierdos y derechos ( y )
t_PARENTESIS_IZQ = r"\("
t_PARENTESIS_DER = r"\)"

# define el token para el final de la linea, que puede ser : o ;
t_FIN_DE_LINEA = r"(:|;)"
# define los tokens para cualquier numero entero del 1 al 9
t_VALENCIA = r"[1-9]"
# t_DIGITO = r"[0-9]"  # define los tokens para cualquier digito del 0 al 9
t_TIPO = r"modelo"  # define el token para la palabra "modelo"
# define los tokens para diferentes tipos de enlaces quimicos
t_ENLACE = r"(-|=|:|::)"
t_ignore = " \t"  # indica que se deben ignorar los espacios en blanco y
# tabulaciones


def t_ASIGNACION(t):  # identifica el token ":="
    r":="
    return t

# identifica el token graficar2d, graficar3d y pesomolecular


def t_OPERACION(t):
    r"(graficar2d|graficar3d|pesomolecular)"
    return t


def t_ELEMENTO_QUIMICO(t):  # define regla para el token elemento quimico
    r"(H|Li|Na|K|Rb|Cs|Fr|Be|Mg|Ca|Sr|Ba|Ra|Sc|Y|Ti|Zr|Hf|Db|V|Nb|Ta|Ji|Cr|Mo\
        |W|Rf|Mn|Tc|Re|Bh|Fe|Ru|Os|Hn|Co|Rh|Ir|Mt|Ni|Pd|Pt|Cu|Ag|Au|Zn|Cd|Hg|B\
        |Al|Ga|In|Ti|Cl|Si|Ge|Sn|Pb|N|P|As|Sb|Bi|O|S|Se|Te|Po|F|C|Br|I|At|He|\
        Ne|Ar|Kr|Xe|Rn)"
    return t

# identifica el token ID pero tambien idetifica el token de palabra reservada


def t_ID(t):
    r"[A-Za-z]+\d*"
    isPR = reserved.get(t.value, "ID")
    # if isPR != "ID":
    #     t.type = "PALABRA_RESERVADA"
    # else:
    #     t.type = isPR
    #     variables[t.value] = ""
    if isPR != "MODELO":
        t.type = isPR
    elif isPR == "MODELO":
        t.type = "TIPO"
    if isPR == "ID":
        variables[t.value] = ""
    return t


def t_newline(t):           # incrementa ek numero de linea
    r'\n+'
    t.lexer.lineno += len(t.value)


def t_COMMENT(t):           # Ignora comentarios
    r'\#.*'
    pass
    # No return value. Token discarded


def t_error(t):             # identifica error lexico
    tokenTable.add_row([tokenNum, "ERROR", t.value[0],
                       t.lineno, t.lexpos, test_file])
    t.lexer.skip(1)


lexer = lex.lex()

tokenNum = 1
if __name__ == "__main__":
    tokenTable.field_names = ["N.", "Token",
                              "Lexema", "Linea", "Posicion", "Programa"]
    reservedWords = PrettyTable()
    reservedWords.field_names = ["Palabra Reservada"]
    symbolsTable = PrettyTable()
    symbolsTable.field_names = ["Variables"]

    with open(test_file, "r") as f:
        with open("output.txt", "w") as o:
            for data in f:
                # data = input("Input data: ")
                lexer.input(data)
                for tok in lexer:
                    tokenTable.add_row(
                        [tokenNum, tok.type, tok.value, tok.lineno, tok.lexpos,
                         test_file])
                    tokenNum += 1
                    # o.write(line+"\n")
            o.write(str(tokenTable))
            line = "\n\nTABLA DE SIMBOLOS"
            o.write(line+"\n")
            for val in variables:
                symbolsTable.add_row([val])
            o.write(str(symbolsTable)+"\n")
            line = "\n\nPALABRAS RESERVADAS"
            o.write(line+"\n")
            for val in reserved:
                reservedWords.add_row([val])
            o.write(str(reservedWords)+"\n")

    with open("output.txt", "r") as f:
        for line in f:
            print(line, end="")
\end{minted}

\label{apendixA}


\chapter{Código de Analizador Sintáctico}

\begin{minted}[breaklines,breakanywhere,linenos,escapeinside=!!]{python}
from lexer import tokens
from lexer import variables
from ply import yacc
import sys

rules = []


def format_expr(p):
    types = [x.type for x in p.slice]
    rule = f"{types[0]} --> "
    for val in types[1:]:
        rule += f"{val} "
    rules.append([rule])
    # print(p.lexspan(1))


start = "s"


def p_s(p):
    '''s : INICIO sentencias FIN'''
    # p[0] = p[2]
    format_expr(p)


def p_sentencias(p):
    '''sentencias : sentencia FIN_DE_LINEA sentencias
                  | sentencia FIN_DE_LINEA'''
    # if (len(p) == 4):
    #     p[0] = p[1] + p[3]
    # elif (len(p) == 3):
    #     p[0] = p[1]
    format_expr(p)


def p_sentencia(p):
    '''sentencia : DEFINA ID COMO TIPO
                  | ID ASIGNACION modelo_molecular
                  | OPERACION PARENTESIS_IZQ ID PARENTESIS_DER'''
    # type1 = p[1].type
    # print(type1)
    format_expr(p)
    # if (len(p) == 4):
    #     print(p[1])
    #     print(p[3])
    #     variables[p[1]] = p[3]


def p_modelo_molecular(p):
    '''modelo_molecular : ELEMENTO_QUIMICO
                        | ELEMENTO_QUIMICO VALENCIA
                        | elemento grupo_funcional
                        | compuesto elemento
                        | compuesto elemento grupo_funcional
                        | compuesto compuesto compuestos'''
    format_expr(p)


def p_compuesto(p):
    '''compuesto : ELEMENTO_QUIMICO
                 | ELEMENTO_QUIMICO VALENCIA
                 | elemento grupo_funcional
                 | elemento grupo_funcional ENLACE
                 | elemento ENLACE'''
    format_expr(p)


def p_compuestos(p):
    '''compuestos : compuesto compuestos
                  | compuesto'''
    format_expr(p)


def p_elemento(p):
    '''elemento : ELEMENTO_QUIMICO
                | ELEMENTO_QUIMICO VALENCIA'''
    format_expr(p)


def p_grupo_funcional(p):
    '''grupo_funcional : grupo_funcional_inferior grupo_funcional_superior
                       | grupo_funcional_superior grupo_funcional_inferior
                       | PARENTESIS_IZQ modelo_grupo_funcional PARENTESIS_DER
                       | COR_IZQ modelo_grupo_funcional COR_DER'''
    format_expr(p)


def p_grupo_funcional_inferior(p):
    '''grupo_funcional_inferior : COR_IZQ modelo_grupo_funcional COR_DER'''
    format_expr(p)


def p_grupo_funcional_superior(p):
    '''grupo_funcional_superior : PARENTESIS_IZQ modelo_grupo_funcional PARENTESIS_DER'''
    format_expr(p)


def p_modelo_grupo_funcional(p):
    '''modelo_grupo_funcional : ENLACE modelo_molecular
                              | ELEMENTO_QUIMICO
                              | ELEMENTO_QUIMICO VALENCIA
                              | elemento grupo_funcional
                              | compuesto elemento
                              | compuesto elemento grupo_funcional
                              | compuesto compuesto compuestos'''
    format_expr(p)


def find_column(input, token):
    line_start = input.rfind('\n', token.lineno, token.lexpos) + 1
    return (token.lexpos - line_start) + 1


try:
    test_file = sys.argv[1]
except IndexError:
    test_file = "test_prog.txt"


def p_error(p):
    if p:
        with open(test_file, "r") as f:
            line = f.read()
        err = f"Error sintactico en la linea {p.lineno}, columna {find_column(line, p)}\
        por {p.type}\n"
        with open("parser_err_out.txt", "a") as f:
            f.write(err)
        print(err)
        parser.errok()
    else:
        print("Error Sintactico en el final del archivo")


parser = yacc.yacc(debug=True)
if __name__ == "__main__":
    while True:
        try:
            s = input('calc > ')
        except EOFError:
            break
        if not s:
            continue
        parser.parse(s)

\end{minted}

\label{apendixB}

\chapter{Casos de Prueba}

\begin{figure}
    \scaledimage{images/corr1.png}
    \caption{Ejemplo de Programa Correcto}
    \label{fig: corr1}
\end{figure}

\begin{figure}
    \scaledimage{images/corr2.png}
    \caption{Ejemplo de Programa Correcto}
    \label{fig: corr2}
\end{figure}

\begin{figure}
    \scaledimage{images/corr3.png}
    \caption{Ejemplo de Programa Correcto}
    \label{fig: corr3}
\end{figure}

\begin{figure}
    \scaledimage{images/incorr1.png}
    \caption{Ejemplo de Programa Incorrecto}
    \label{fig: incorr1}
\end{figure}

\begin{figure}
    \scaledimage{images/incorr2.png}
    \caption{Ejemplo de Programa Incorrecto}
    \label{fig: incorr2}
\end{figure}

\begin{figure}
    \scaledimage{images/incorr3.png}
    \caption{Ejemplo de Programa Incorrecto}
    \label{fig: incorr3}
\end{figure}

\label{appendixC}

\listoffigures

\listoftables


\bibliographystyle{apacite}
\bibliography{References}
\printglossary[type=\acronymtype]
\end{document}


